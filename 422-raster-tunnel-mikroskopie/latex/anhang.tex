\chapter{Anhang}

\section{Kalibrationsaufnahmen des Netzchens}

\jafpsh{netz_gold_scaled.pdf}{USB-Mikroskopaufnahme des Kalibrationsnetzchens der Ausmaße $\SI{254}{\micro\meter}\times \SI{63.5}{\micro\meter}$ für die erste Goldprobe. Maßstab eingefügt mit GIMP.}{0.4}

\jafpsh{netz_gold2_scaled.pdf}{USB-Mikroskopaufnahme des Kalibrationsnetzchens für die zweite Goldprobe. Maßstab aus den bekannten Maßen des Netzchens bestimmt und mit GIMP eingefügt.}{0.4}

\jafpsh{netz_hopg_scaled.pdf}{USB-Mikroskopaufnahme des Kalibrationsnetzchens für die HOPG-Probe. Maßstab aus den Maßen des Netzchens mit GIMP eingefügt.}{0.4}
\jafpsh{netz_mos2_scaled.pdf}{USB-Mikroskopaufnahme des Kalibrationsnetzchens für die \moszwei-Probe. Maßstab aus den Maßen des Netzchens mit GIMP eingefügt.}{0.4}

\section{RTM-Aufnahmen der ersten Goldprobe} \label{sec:anhanggoldrtm}

\jafpsh{res/gold_126_s4.pdf}{RTM-Aufnahme ($\SI{200}{\nano\meter}$, z-Darstellung) von Goldprobe 1 mit Spitze \ref{fig:spitze_4_scaled.pdf}. Aufnahmeparameter: $P=1000, I=2000, I_\text{Setpoint}=\SI{1}{\nano\A}$, $v_R=\SI{200,2}{\nano\meter\per\s}$.}{0.45}

\jafpsh{res/gold_127_s4_z.pdf}{RTM-Aufnahme ($\SI{200}{\nano\meter}$, z-Darstellung) von Goldprobe 1 mit Spitze \ref{fig:spitze_4_scaled.pdf}. Aufnahmeparameter: $P=1000, I=2000, I_\text{Setpoint}=\SI{1}{\nano\A}$, $v_R=\SI{200,2}{\nano\meter\per\s}$.}{0.45}


\section{RTM-Aufnahmen der HOPG-Probe}

\jafpsh{res/hopg_165}{RTM-Aufnahme ($\SI{4.4}{\nano\meter}$, Stromdarstellung), $v_R = \SI{157.19}{\nano\meter\per\second}$, $I_\text{Setpoint} = \SI{751}{\nano\ampere}$, $P=0$, $I=4$}{0.4}

\jafpsh{res/hopg_166_lattice}{Gleiche Aufnahme wie Abbildung \ref{fig:res/hopg_166_with_lines}, mit Kristallgitter eingezeichnet.}{0.4}

\section{RTM-Aufnahmen der \moszwei-Probe}

\jafpsh{res/mos2_169}{RTM-Aufnahme ($\SI{68}{\nano\meter}$, z-Darstellung), $v_R = \SI{340.0}{\nano\meter\per\second}$, $I_\text{Setpoint} = \SI{1}{\nano\ampere}$, $P=1000$, $I=2000$}{0.4}
\jafpsh{res/mos2_171}{RTM-Aufnahme ($\SI{9.82}{\nano\meter}$, z-Darstellung), $v_R = \SI{49.1}{\nano\meter\per\second}$, $I_\text{Setpoint} = \SI{1}{\nano\ampere}$, $P=1000$, $I=2000$}{0.4}

\jafpsh{res/mos2_197_raw}{Gleiche Aufnahme wie \ref{fig:res/mos2_197}, allerdings ohne Korrektur der Schieflage.}{0.4}
