\chapter{\moszwei}
Zuletzt wird noch eine \moszwei Probe untersucht. Molybdän-Sulfid (\moszwei) ist eine Kristallstruktur aus Molybdän und Schwefel, welche ähnlich wie HOPG eine schichtartige Struktur, bei welcher die einzelnen Schichten nur durch Van-der-Vaals Kräfte zusammengehalten werden. Die Kristallstruktur einer Schicht ist in der Regel hexagonal. Hierbei sind an der Oberfläche Schwefelatome welche gleichseitige Dreiecke mit einem Abstand von $d = \SI{0.31604}{nm}$ die dominant messbare Struktur\cite{mos2structure}.

\section{Durchführung}
Das Messverfahren ist grundsätzlich gleich dem in Abschnitt \ref{sec:hopg_messung} beschriebenen. Zunächst wird die Probe mit einem Tesastreifen präperiert um eine saubere und intakte Fläche zum Messen zu haben. Dann wird ein übersichtsbild im CCM erstellt und eine ausreichend glatte Stelle gesucht. Ein geeigneter Ausschnitt wird gewählt und das Prozedere wiederholt bis ein glatter Ausschnitt mit wenigen Nanometern Kantenlänge gefunden ist. Aufnahmen von den Annäherungsschritten sind im Anhang zu finden \todo{refrence}.
Es werden Messungen eines kleinen, ebenen Bereichs im CHM angefertigt, zu sehen in Abbildung \ref{fig:res/mos2_197}. In diese Messung werden nun mit Gwyddion \cite{gwyddion} Gitterabstände eingezeichnet und vermessen. Diese Messung sind in Abbildung \ref{fig:res/mos2_lattice_197} und Tabelle \ref{tab:mos_lattice} zu finden.

Die Abstände wurde für jeweils drei Atome gemessen, so dass die Unsicherheit insgesamt kleiner wird. Die drei Richtungen werden wieder seperat behandelt. Man findet dann die mittleren Abstände und Winkel, aufgelistet in Tabelle \ref{tab:mos_lattice_avg}.

\jafps{res/mos2_197}{\todo{caption}}{0.4}
\jafps{res/mos2_lattice_197}{Ausschnitt von Abbildung \ref{fig:res/mos2_197}, mit eingezeichnetem Gitter}{0.4}

\begin{table}[H]
\centering
\begin{tabular}{|c|c|c|}
\hline
Index & $\varphi / \unit{\degree}$ & $L / \unit{\nano\meter}$\\
\hline
1 & $ \num{61(2)}$ & $ \num{0.85(3)} $ \\
2 & $ \num{63(2)}$ & $ \num{0.85(3)} $ \\
3 & $ \num{61(2)}$ & $ \num{0.83(3)} $ \\
4 & $ \num{61(2)}$ & $ \num{0.85(3)} $ \\
\hline
5 & $ \num{135(2)}$ & $ \num{0.89(3)} $ \\
6 & $ \num{137(2)}$ & $ \num{0.89(3)} $ \\
7 & $ \num{-45(2)}$ & $ \num{0.90(3)} $ \\
\hline
8 & $ \num{-173(2)}$ & $ \num{0.98(3)} $ \\
9 & $ \num{7(2)}$ & $\num{0.96(3)}$ \\
10 & $ \num{5(2)}$ & $ \num{0.93(3)}$ \\
\hline
\end{tabular}
\caption{Längen und Winkel der in Abbildung \ref{fig:res/mos2_lattice_197} eingezeichneten Linien. Die Unsicherheit ergibt sich aus der Ableseungenauigkeit beim suchen der Grenzen der einzelnen Atome. Die Linien wurden zum besseren Mitteln jeweils über den dreifachen Atomabstand gezogen.}
\label{tab:mos_lattice}
\end{table}


\begin{table}[H]
\centering
\begin{tabular}{|c|c|c|c|}
\hline
Richtung & Index & $\varphi / \unit{\degree}$ & $L / \unit{\nano\meter}$\\
\hline
A & $1 - 4$ & $\num{62(1)}$ & $\num{0.28(1)}$ \\
B & $5 - 7$ & $\num{136(1)}$ & $\num{0.30(1)}$ \\
C & $8-10$ & $\num{6(1)}$ & $\num{0.32(1)}$ \\
\hline
\end{tabular}
\caption{Mittelwerte der Daten aus Tabelle \ref{tab:mos_lattice}.}
\label{tab:mos_lattice_avg}
\end{table}

Man findet weiterhin die Winkel zwischen den Gruppen als $\Delta\varphi_{A, B} = |\varphi_A - \varphi_B| = $
