\chapter{Fazit}
Die Darstellung von aufgedampften Goldkugeln ist insgesamt gut gelungen. Zunächst musste sowohl das Spitzen reissen als auch die Nutzung des RTM geübt werden. Nachdem einmal eine gut funktionierende Spitze gerissen worden war (siehe Abbildung \ref{fig:spitze_4_scaled.pdf}) ließen sich beide Golproben darstellen. Die Gold auf Saphir Probe konnte gut dargestellt werden mit klar erkennbaren Goldflocken (vergleiche Abbildungen \ref{fig:res/gold_126_s4.pdf}, \ref{fig:res/gold_129_s4.pdf} und \ref{fig:res/gold_126_s4.pdf}).
