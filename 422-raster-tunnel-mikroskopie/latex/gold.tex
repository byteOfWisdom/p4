\chapter{Untersuchung von Gold im RTM}

Um die Funktionsweise des RTM im Bereich größer der atomaren Auflösung von Graphit-Gittern zu überprüfen, wird zunächst die Oberflächentopologie einer mit Gold bedampften Silizium- oder Saphir-Probe untersucht. Welche der beiden Probenarten untersucht wurde, kann anhand des erhaltenden RTM-Bilds festgestellt werden, die Proben waren jedoch nicht beschriftet. Es werden die beiden Goldproben (Probe 1: Nummer 16, Probe 2: Nummer 14) nacheinander mit dem RTM untersucht.

\section{Theoretische Überlegungen} \label{sec:theoriegold}
Es wird nicht direkt Gold untersucht, welches aufgrund seiner elektrisch relativ flachen kristallinen Struktur isoliert zu nur wenige kontrastreichen Aufnahmen führen würde \cite[59-60]{STMrohrer}. Stattdessen werden Proben von Silizium bzw. Saphir verwendet, auf welche Gold aufgedampft wurde \cite[13]{Versuchsanleitung422}.
Auf der glatten Si- oder Saphir-Oberfläche sind daher kleine Goldkügelchen ohne festes Muster angeordnet. Sie sollten so bei supra-atomaren Auflösungen des Mikroskops gut sichtbar sein. Durch das  Erhitzen muss damit gerechnet werden, dass die Goldkügelchen in ihrer Oberfläche stufenförmige Strukturen aufweisen anstatt vollständig glatt zu sein. Die Ausprägung der Stufen ist stark von der erreichten Temperatur während des Erhitzens abhängig \cite[59]{STMrohrer}. Die vermessenen Proben bestehen aus jeweils zwei verschiedenen Atomsorten (Gold und Silizium bzw. Gold und Saphir). Die Austrittsarbeiten der Elemente unterscheiden sich\cite[Anhang A]{crystalworkfuncs}, weshalb nicht unbedingt direkt auf ein räumliches Höhenprofil geschlossen werden kann\cite[3]{Versuchsanleitung422}. Es wird davon ausgegangen, dass die Oberfläche im wesentlichen komplett mit Goldkugeln bedeckt ist oder deren Höhe (in der Größenordnung von $\mathcal{O}(\unit{\nano\meter})$) zumindest so groß ist, dass die Messung in guter Näherung die Topologie der Oberfläche widerspiegelt.


\section{Durchführung}

\subsection{Verwendete RTM-Spitze und Proben}

Zunächst werden die beiden Goldproben sowie die verwendete Spitze mit dem USB-Mikroskop dokumentiert. Durch letzteres kann zwar keine quantitative Einschätzung der Qualität der Spitze liefern (dafür müsste das Mikroskop atomare Auflösung haben, da dies die ideale Dicke der Spitze ist), kann aber einen guten Anhaltspunkt liefern, ob der Reißvorgang bei der Spitzenpräperation wie erwünscht eine schräge Kante entstanden ist\footnote{Eine schräge Kante erhöht dich Chance auf eine Atomar dünne Spitze, kann aber keine konkrete Aussage treffen}. Es werden zunächst mehrere Spitzen ausprobiert, gute RTM-Aufnahmen liefert jedoch nur eine der Spitzen. Diese ist in Abb. \ref{fig:spitze_4_scaled.pdf} in einer USB-Mikroskop-Aufnahme abgebildet ist.

\jafps{spitze_4_scaled.pdf}{USB-Mikroskopaufnahme der für die RTM-Aufnahmen genutzten Spitze. Der eingezeichnete Maßstab ergibt sich aus dem Durchmesser des genutzten Platin-Iridium-Draht von $r=\SI{0.3}{\milli\meter}$ \cite[8]{Versuchsanleitung422}. Zum Einfügen des Maßstabs wurde das Bildbearbeitungsprogramm GIMP \cite{gimp} genutzt.}{0.4}

Um die USB-Mikroskopaufnahmen der ersten Goldprobe mit einer Skala versehen zu können, wird jeweils ein Kalibrationsnetzchen mit bekannten Dimensionen in der gleichen Brennebene wie die Aufnahmen der Probe aufgenommen. Aus der so berechenbaren Bildskala wird eine Skala für die Probenaufnahme bestimmt.
Analog wird im Folgenden auch für die USB-Mikroskopaufnahmen der weiteren Proben (zweite Goldprobe, HOPG, \moszwei), jeweils mit individuellen Kalibrationsaufnahmen des Netzchens, vorgegangen.
Das Kalibrationsnetz besitzt ein Netz aus Divisionen der Größe $1~\unit{Zoll}/100$ in vertikaler Richtung und $1~\text{Zoll}/400$ in horizontaler Richtung, siehe Abbildung \ref{fig:netz_gold_scaled.pdf}. Aus der Entsprechung $1~\text{Zoll}=\SI{25.4}{\milli\meter}$\cite[11]{usunits} ergibt sich als Unterteilung $\SI{254}{\micro\meter}\times \SI{63.5}{\micro\meter}$ (vertikale $\times$ horizontale Division).
Die Aufnahmen für die erste Goldprobe sind in Abbildungen \ref{fig:netz_gold_scaled.pdf} und \ref{fig:gold_scaled.pdf} dargestellt.
\jafps{gold_scaled.pdf}{USB-Mikroskopaufnahme der ersten Goldprobe. Maßstab mithilfe der Kalibrationsaufnahme \ref{fig:netz_gold_scaled.pdf} bestimmt.}{0.4}
Auf Abbildung \ref{fig:gold_scaled.pdf} sind deutliche Oberflächenunreinheiten erkennbar, die RTM-Aufnahmen haben aber einen deutlich kleineren Bildausschnitt (quadratische Abmessung von $\SI{200}{\nano\meter}$), sodass nicht davon ausgegangen werden kann, dass die dort sichtbaren Strukturen diesen mit dem optischen Mikroskop sichtbaren Kratzern entsprechen.
Die korrespondierenden Aufnahmen für die zweite Goldprobe sind in Abbildungen \ref{fig:netz_gold2_scaled.pdf} und \ref{fig:gold2_scaled.pdf} dargestellt.
\jafps{gold2_scaled.pdf}{USB-Mikroskopaufnahme der zweiten Goldprobe, Maßstab mithilfe der Kalibrationsaufnahme \ref{fig:netz_gold2_scaled.pdf} bestimmt.}{0.4}

\subsection{Einstellung des RTMs}

Da die Goldproben durch die relativ großen Goldkügelchen auf den flachen Basismaterialien hohe Korrugation aufweisen (siehe Abschnitt \ref{sec:theoriegold}), wird im Folgenden das RTM immer im CCM betrieben. Die Parameter des Regelkreises waren dabei auf $P=1000, I=2000$ eingestellt. Eine genauere Auflösung, wie sie durch den CHM erreicht hätte werden können, hätte zu keinen zusätzlichen Informationen geführt. Es ist zu beachten, dass für die beiden Proben, die sich in den Basismaterialien unterscheiden, verschiedene Goldkügelchen-Verteilungen zu erwarten sind, was auch in unterschiedlichen Strukturen auf dem RTM-Aufnahmen sichtbar werden sollte \cite[13]{Versuchsanleitung422}.

\section{Messergebnisse}

Zur Darstellung und nachträglichen Bearbeitung der Daten wird das Programm gwyddion \cite{gwyddion} genutzt. Da die Proben teilweise noch eine Restverkippung gegenüber der x-y-Ebene der Spitzenrasterung sowie eine sichtbare Streifencharakteristik der Rohdaten aufwiesen, wird für alle hiernach aufgeführten RTM-Aufnahmen (außer explizit anders angegeben) standardmäßig die folgende Bildarbeitung mithilfe der Funktionalitäten von gwyddion durchgeführt \cite{docs}:

1. \textit{Plane Level}-Funktionalität: Die Datenpunkte werden analysiert, eine gemeinsame Ebene wird aus ihnen berechnet und von den Datenpunkten einzeln subtrahiert. Dies führt zu einem Ausgleich einer möglichen Verkippung der Probe.

2. \textit{Remove Scars}-Funktionalität: Automatisches Finden und Ausgleich von Liniendefekten in den Datenpunkten, die parallel zur x-Rasterachse auftreten. Sie sind in der Regel durch lokale Regelkreisfehler verursacht und werden durch Interpolation benachbarter, defektfreier Linien ausgeglichen.

Die so erhaltenen RTM-Aufnahmen zeigten besseren Kontrast und weniger Bildfehler (Linienfehler), ohne dass sich die zu untersuchenden Strukturen (in diesem Abschnitt Verteilung der hellen und dunklen Stellen, in Abschnitten zu den Proben mit HOPG und \moszwei die geometrischen Abmessungen der atomaren Aufnahmen) verändern, wodurch die genaue Analyse dieser Strukturen einfacher wurde.

In den Abbildungen \ref{fig:res/gold_126_s4.pdf},\ref{fig:res/gold_127_s4_z.pdf} und \ref{fig:res/gold_129_s4.pdf} sind die RTM-Aufnahmen der ersten Goldprobe (Nummer 16) dargestellt. In Abbildung \ref{fig:res/gold2_130_s4.pdf} ist zusätzlich noch eine RTM-Aufnahme der zweiten Goldprobe (Nummer 14) dargestellt.
Zur besseren Übersichtlichkeit ist von der ersten Probe nur eins der Bilder hier dargestellt, die anderen beiden sind im Anhang (Abschnitt \ref{sec:anhanggoldrtm}) zu finden.
Die Aufnahmen wurden wie oben erläutert alle im CCM aufgenommen.

\jafps{res/gold_129_s4.pdf}{RTM-Aufnahme (Bildgröße: $\SI{200}{\nano\meter}$, z-Darstellung) von Goldprobe 1 mit Spitze \ref{fig:spitze_4_scaled.pdf}. Aufnahmeparameter: $P=1000, I=2000, I_\text{Setpoint}=\SI{1}{\nano\A}$, Rastergeschwindigkeit $v_R=\SI{250,3}{\nano\meter\per\s}$.}{0.45}

\jafps{res/gold2_130_s4.pdf}{RTM-Aufnahme ($\SI{200}{\nano\meter}$, z-Darstellung) von Goldprobe 2 mit Spitze \ref{fig:spitze_4_scaled.pdf}. Aufnahmeparameter: $P=1000, I=2000, I_\text{Setpoint}=\SI{1}{\nano\A}$, $v_R=\SI{200,2}{\nano\meter\per\s}$.}{0.4}

Die Aufnahmen der Goldproben weisen wie erwartet eine raue Struktur auf. Es sind auf den Aufnahmen der ersten Probe wolkenförmige, nicht ganz runde Strukturen mit für sich jeweils recht ebener Oberfläche zu erkennen. Es ist hier immer das Höhenbild dargestellt. Man erkennt verschieden große Goldflocken. In Abbildung \ref{fig:res/gold_126_s4.pdf} gibt auch noch kleinere Bereiche mit helleren (höheren) Flecken, dies könnte entweder von kleineren Goldflocken oder (ggf. durch das Erhitzen hervorgerufenen) Stufen (siehe Abschnitt \ref{sec:theoriegold} stammen. Messungenauigkeiten durch eine nicht-ideale, d.h. nicht einatomige, Spitze sind allerdings hier auch denkbar als Ursache.
Insgesamt ist auf den Aufnahmen der ersten Goldprobe eine Struktur zu erkennen, die darauf hinweist, dass die Probe aus Gold und Saphir bestand. Vergleiche hierzu auch die Referenzaufnahme zu Gold auf Saphir in \cite[13]{Versuchsanleitung422}, die eine ähnliche Struktur aufweist.

Die Aufnahme der zweiten Probe weist auffallend im unteren Bildteil eine Stufe auf. Hierbei handelt es sich vermutlich um einen Messfehler. Mögliche Ursachen für diesen sind ein nicht optimal eingestellter Regelkreis, ungünstig gesetzte Werte für den Setpoint (Vergleichswert für den Tunnelstrom, der durch Ausgleich der Höhe erreicht werden soll) oder die Tunnelspannung oder gegebenenfalls auch ein Staubkorn auf der Probe. Da diese Bildfehler nur an einer Stelle und nur bei der zweiten Goldprobe aufgetreten sind ist letzteres hierbei die wahrscheinlichere
Ursache.
Daher wird dieser Bereich der Probe nicht weiter beachtet und nur der obere Teil untersucht. Hier ist eine deutlich homogenere Struktur zu erkennen als bei den Aufnahmen der ersten Goldprobe. Dies deutet darauf hin dass es sich hier um eine Aufnahme von Gold auf Silizium handelt. Die Kügelchen konnten hier nicht einzeln aufgelöst werden. Für eine Verbesserung der Auflösung wäre eine CHM Aufnahme eines kleinen Teilstücks des Bildausschnitts sinnvoll. Eine Aufnahme im CHM führt in der Regel zu besserer Auflösung der tatsächlichen Oberflächenstruktur, da Resonanzen und Störeffekte des Regelkreises so stark reduziert werden können. Um die eingebaute Spitze zu schonen, wurde für diese Probe jedoch auf eine solche Aufnahme verzichtet. Die gemachten Bilder, insbesondere von der ersten Probe, zeigen insgesamt ein gut funktionierendes RTM.
