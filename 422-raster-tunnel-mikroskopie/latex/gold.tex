\chapter{Untersuchung von Gold im RTM}

Um die Funktionsweise des RTM im Bereich größer der atomaren Auflösung von Graphit-Gittern zu überprüfen, wird zunächst die Oberflächentopologie einer mit Gold bedampften Silizium- oder Saphir-Probe untersucht. Welche der beiden Probenarten untersucht wurde, kann anhand des erhaltenden RTM-Bilds erschlossen werden, die Proben waren jedoch nicht beschriftet. Es werden die beiden Goldproben (Probe 1: Nummer 16, Probe 2: Nummer 14) nacheinander mit dem RTF untersucht.

\section{Theoretische Überlegungen}\label{sec:theoriegold}
Es wird nicht direkt Gold untersucht, welches aufgrund seiner elektrisch relativ flachen kristallinen Struktur isoliert zu nur wenige kontrastreichen Aufnahmen führen würde \cite[59-60]{STMrohrer}. Stattdessen werden Proben von Silizium bzw. Saphir verwendet, auf welche Gold aufgedampft wurde \cite[13]{Versuchsanleitung422}.
Auf der glatten Si- oder Saphir-Oberfläche sind daher kleine Goldkügelchen ohne festes Muster angeordnet. Sie sollten so bei supra-atomaren Auflösungen des Mikroskops gut sichtbar sein. Eine atomare Auflösung der RTM-Aufnahmen ist bei diesen Proben auch nicht sinnvoll, da Gold, wie genannt, sehr gleichmäßige Elektronenverteilungen aufweist und so keine deutlichen Strukturen innerhalb des Kristallgitters erkennbar sein würden. Durch den Erhitzungsprozess kann allerdings damit gerechnet werden, dass die Goldkügelchen in ihrer Oberfläche stufenförmige Strukturen aufweisen anstatt vollständig glatt zu sein, allerdings ist die Ausprägung der Stufen stark von der erreichten Temperatur während des Erhitzens abhängig \cite[59]{STMrohrer}. Außerdem ist zu beachten, dass die vermessenen Proben aus jeweils zwei verschiedenen Atomsorten bestehen (Gold und Silizium bzw. Gold und Saphir), die zu verschiedenen wirkenden Austrittsarbeiten abhängig von der Position der Spitze auf der Proben führen, kann aus dem Tunnelstrom nicht direkt auf die Oberflächentopologie geschlossen werden. Es wird ein Bild der Austrittsarbeiten gemessen, nicht der tatsächlichen physischen Oberflächenbeschaffenheit \cite[3]{Versuchsanleitung422}. Da bei den Goldprobe aufgrund der Herstellungsart jedoch relativ große Kügelchen des Golds der Ausdehnungsgrößenordnung $\mathcal{O}(\unit{nm})$ (siehe auch \cite[13]{Versuchsanleitung422} für de Größenordnung) an den meisten Stellen der Probe mit lokal relativ ebener Austrittsarbeitsverteilung zu erwarten sind (welche bei Gold zudem noch deutlich größer ist als die von Si und Saphir \cite[Anhang A]{crystalworkfuncs}), bilden die erhaltenen RTM-Aufnahmen in guter Näherung die topologische Beschaffenheit der Proben ab.

\section{Durchführung}

\subsection{Verwendete RTM-Spitze und Proben}

Zunächst werden die beiden Goldproben sowie die verwendete Spitze mit dem USB-Mikroskop dokumentiert. Durch letzteres kann zwar keine quantitative Einschätzung der Qualität der Spitze liefern (dafür müsste das Mikroskop atomare Auflösung haben, da dies die ideale Dicke der Spitze ist), kann aber einen guten Anhaltspunkt liefern, ob der Reißvorgang bei der Spitzenpräperation wie erwünscht eine schräge Kante, bei der eine spitzere Spitze wahrscheinlicher ist, entstanden ist oder ob der Draht planar und eher stumpf abgerissen wurde. Im Versuchsdurchlauf wurden zunächst mehrere Spitzen ausprobiert, gute RTM-Aufnahmen lieferte jedoch nur eine der Spitzen, die in Abb. \ref{fig:spitze_4_scaled.pdf} in einer USB-Mikroskop-Aufnahme abgebildet ist.
\jafps{spitze_4_scaled.pdf}{USB-Mikroskopaufnahme der für die RTM-Aufnahmen genutzten Spitze. Der eingezeichnete Maßstab ergibt sich aus dem Durchmesser des genutzten Platin-Iridium-Draht von $r=\SI{0.3}{\milli\meter}$ \cite[8]{Versuchsanleitung422}. Zum Einfügen des Maßstabs wurde das Bildbearbeitungsprogramm GIMP \cite{gimp} genutzt.}{0.4}
Um die USB-Mikroskopaufnahmen der ersten Goldprobe mit einer Skala versehen zu können, wird jeweils ein Kalibrationsnetzchen mit bekannten Dimensionen in der gleichen Brennebene wie die Aufnahmen der Probe aufgenommen. Aus der so berechenbaren Bildskala kann eine Skala für die Probenaufnahme bestimmt werden.
Analog wird im Folgenden auch für die USB-Mikroskopaufnahmen der weiteren Proben (zweite Goldprobe, HOPG, \moszwei), jeweils mit individuellen Kalibrationsaufnahmen des Netzchens, vorgegangen.
Das Kalibrationsnetz besitzt ein Netz aus Divisionen der Größe $1~\unit{Inch}/100$ in vertikaler Richtung und $1~\text{Inch}/400$ in horizontaler Richtung, siehe Abbildung \ref{fig:netz_gold_scaled.pdf}. Aus der Entsprechung $1~\text{Inch}=\SI{25.4}{\milli\meter}$\cite[11]{usunits} ergibt sich als Unterteilung $\SI{254}{\micro\meter}\times \SI{63.5}{\micro\meter}$ (vertikale $\times$ horizontale Division).
Die Aufnahmen für die erste Goldprobe sind in Abbildungen \ref{fig:netz_gold_scaled.pdf} und \ref{fig:gold_scaled.pdf} dargestellt.
\jafps{gold_scaled.pdf}{USB-Mikroskopaufnahme der ersten Goldprobe. Maßstab mithilfe der Kalibrationsaufnahme \ref{fig:netz_gold_scaled.pdf} bestimmt.}{0.4}
Auf Abbildung \ref{fig:gold_scaled.pdf} sind deutliche Oberflächenunreinheiten erkennbar, die RTM-Aufnahmen haben aber einen deutlich kleineren Bildausschnitt (quadratische Abmessung von $\SI{200}{\nano\meter}$), sodass nicht davon ausgegangen werden kann, dass die dort sichtbaren Strukturen diesen mit dem optischen Mikroskop sichtbaren Kratzern entsprechen. \todo{vergleich rtm vs optische korrosion}
Die korrespondierenden Aufnahmen für die zweite Goldprobe sind in Abbildungen \ref{fig:netz_gold2_scaled.pdf} und \ref{fig:gold2_scaled.pdf} dargestellt.
\jafps{gold2_scaled.pdf}{USB-Mikroskopaufnahme der zweiten Goldprobe, Maßstab mithilfe der Kalibrationsaufnahme \ref{fig:netz_gold2_scaled.pdf} bestimmt.}{0.4}

\subsection{Einstellung des RTMs}

Da die Goldproben durch die relativ großen Goldkügelchen auf den flachen Basismaterialien hohe Korrugation aufweisen (siehe Abschnitt \ref{sec:theoriegold}), wird im Folgenden das RTM immer im \textit{constant current}-Modus betrieben. Die Parameter des Regelkreises waren dabei auf $P=1000, I=2000$ eingestellt. Eine Aufnahme im \textit{constant height}-Modus hätte eine ausreichend ebene Fläche erfordert, was aber bei den Goldproben zu keinen interessanten Ergebnissen geführt hätte: Es war zu erwarten, dass die Goldkügelchen, die sich physisch erhöht auf der Oberfläche des Basismaterials befinden, auch im Höhenbild (Verlauf der Spitzenhöhe als \textit{z-Höhe}) als größere, für sich relativ homogene Strukturen erhöht erscheinen würde, während die Lücken zwischen ihnen aufgrund ihrer deutlich geringere Elektronenaufenthaltswahrscheinlichkeiten (also niedrigeren Tunnelstrom) zu niedrigeren Spitzenposition (z-Werte) führen würden. Die relativ (zur atomaren Größenordnung $\mathcal{O}(\SI{e-10}{\meter}=\unit{\AA})$) großen Kügelchen sollten in den Aufnahmen deutlich als quer über das Basismaterial verteilt sichtbar werden. Eine genauere Auflösung, wie sie durch den \textit{constant height}-Modus erreicht hätte werden können, hätte zu keinen zusätzlichen Informationen geführt. Es ist zu beachten, dass für die beiden Proben, die sich in den Basismaterialien unterscheiden, verschiedene Goldkügelchen-Verteilungen zu erwarten waren, was auch in unterschiedlichen Strukturen auf dem RTM-Aufnahmen sichtbar werden sollte \cite[13]{Versuchsanleitung422}.

\section{Messergebnisse}

In den Abbildungen \ref{fig:res/gold_126_s4.pdf},\ref{fig:res/gold_127_s4_z.pdf} und \ref{fig:res/gold_129_s4.pdf} sind die RTM-Aufnahmen der ersten Goldprobe (Nummer 16) dargestellt. In Abbildung \ref{fig:res/gold2_130_s4.pdf} ist zusätzlich noch eine RTM-Aufnahme der zweiten Goldprobe (Nummer 14) dargestellt.
Zur besseren Übersichtlichkeit ist von der ersten Probe nur eins der Bilder hier dargestellt, die anderen beiden sind im Anhang (Abschnitt \ref{sec:anhanggoldrtm}) zu finden.
Die Aufnahmen wurden wie oben erläutert alle im \textit{constant-current}-Modus aufgenommen.

\jafps{res/gold_129_s4.pdf}{RTM-Aufnahme (Bildgröße: $\SI{200}{\nano\meter}$, z-Darstellung) von Goldprobe 1 mit Spitze \ref{fig:spitze_4_scaled.pdf}. Aufnahmeparameter: $P=1000, I=2000, I_\text{Setpoint}=\SI{1}{\nano\A}$, $v_R=\SI{250,3}{\nano\meter\per\s}$.}{0.45}

\jafps{res/gold2_130_s4.pdf}{RTM-Aufnahme ($\SI{200}{\nano\meter}$, z-Darstellung) von Goldprobe 2 mit Spitze \ref{fig:spitze_4_scaled.pdf}. Aufnahmeparameter: $P=1000, I=2000, I_\text{Setpoint}=\SI{1}{\nano\A}$, $v_R=\SI{200,2}{\nano\meter\per\s}$.}{0.4}

Die Aufnahmen der Goldproben weisen wie erwartet eine raue Struktur auf. Es sind auf den Aufnahmen der ersten Probe wolkenförmige, nicht ganz runde Strukturen mit für sich jeweils recht ebener Oberfläche zu erkennen. Es ist hier immer das Höhenbild, also die Abbildung der z-Position der Mikroskopspitze zur Beibehaltung eines Tunnelstroms gleich dem Referenzwert $I_\text{Setpoint}$, abgebildet, die Farbskala der Aufnahmen entspricht dabei bei dunklerer Farbe einem kleineren z-Abstand der Spitze von der Probenoberfläche am jeweiligen Pixel-Ort. Auf den Aufnahme der ersten Probe sind kleine, dunklere Bereiche zu erkennen, bei denen die Austrittsarbeit niedriger war als bei den helleren Bereichen (welche vermutlich ungefähr den Goldkügelchen auf der Probenoberfläche entsprechen) oder die Probenoberfläche physisch tiefer war als bei den helleren (Gold-)Gebieten. Es kann auch eine Kombination der beiden Faktoren vorhanden sein. Die dunkleren Flecken entsprechen also wahrscheinlich den freien Räumen zwischen den aufgedampften Goldkügelchen. Es gibt auch noch kleinere Bereiche mit helleren (höheren) Flecken, dies könnte entweder von kleineren Goldflocken oder (ggf. durch das Erhitzen hervorgerufenen) Stufen (siehe Abschnitt \ref{sec:theoriegold) stammen. Messungenauigkeiten durch eine nicht-ideale, d.h. nicht einatomige, Spitze sind allerdings hier auch denkbar als Ursache. Auch die auf den USB-Mikroskopaufnahmen optisch sichtbaren Kratzer in der Probenoberfläche sowie mögliche weitere kleinere Kratzer, die dieses Mikroskop nicht mehr auflösen kann, sind als Ursache für die flockenartigen Bereiche möglich.
Insgesamt ist auf den Aufnahmen der ersten Goldprobe eine Struktur zu erkennen, die darauf hinweist, dass die Probe aus Gold und Saphir bestand. Vergleiche hierzu auch die Referenzaufnahme zu Gold auf Saphir in \cite[13]{Versuchsanleitung422}, die eine ähnliche Struktur aufweist.

Die Aufnahme der zweiten Probe weist auffallend im unteren Bildteil eine Stufe auf. Da die Aufnahme immer von unten nach oben linienweise stattfindet, ist hier davon auszugehen, dass sich in diesem Bereich der Regelkreis und die Piezoelemente noch nicht der Struktur der Probe angepasst haben, insbesondere, da Piezokristalle eine Hysterese aufweisen
\cite{piezoelements} und davor die erste Goldprobe ausgemessen wurde, die eine sichtbar andere Struktur aufweist, auf welche die Steuerung vermutlich noch geringfügig angepasst war. Daher wird dieser Bereich der Probe nicht weiter beachtet und nur der obere Teil untersucht. Hier ist eine deutlich homogenere Struktur zu erkennen als bei den Aufnahmen der ersten Goldprobe. Es sind aber strangförmige, helle (hohe) Bereiche zu erkennen, zwischen denen Bereiche dunklerer Farbe, also niedrigerer z-Position, liegen. Diese Struktur ähnelt der Referenzaufnahme für Gold auf Silizium in \cite[13]{Versuchsanleitung422}, jedoch hat die Aufnahme \ref{fig:res/gold2_130_s4.pdf} einen größeren Bildausschnitt als das Referenzbild. Die Strukturen in diesem sollten also so nicht in der gemachten Aufnahme sichtbar sein. Die Strangstruktur der Aufnahme weist trotzdem darauf hin, dass es sich hier um eine Aufnahme von Gold auf Silizium handelt. Für eine Verbesserung der Auflösung, sodass sichergestellt werden kann, dass es sich tatsächlich um ähnliche Strukturen wie in der Referenzaufnahme handelt, wäre eine \textit{constant-height}-Aufnahme eines kleinen, anscheinend ebenen Teilstücks des Bildausschnitts sinnvoll. Eine Aufnahme im \textit{constant-height}-Modus führt in der Regel zu besserer Auflösung der tatsächlichen Oberflächenstruktur, da Resonanzen und Störeffekte des Regelkreises so stark reduziert werden können. Um die eingebaute Spitze zu schonen, wurde für diese Probe jedoch auf eine solche Aufnahme verzichtet. Die gemachten Bilder, insbesondere von der ersten Probe, zeigen insgesamt ein gut funktionierendes RTM.
