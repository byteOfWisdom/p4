\chapter{Highly Ordered Pyrolitic Graphite (HOPG)}
\todo{Kristallstruktur}
\todo{Bindungen zwischen Ebenen} 
\todo{Messbare Elektronen}

\section{Durchführung}
Die HOPG Probe wird zunächst unter dem USB Mikroskop dokumentiert (siehe \ref{fig:HOPG_optisch})
\todo{bild mit skala}

Anschließend wird die Probe vorbereitet in dem ein Stück Tesa Film auf die Probe geklebt und dann abgezogen wurde. Dies sorgt dafür, dass eine Frische, unverschmutzte Graphitschicht freigelegt wird.
Es wird zunächst im CCM ein $\SI{200}{\nano\meter} \times \SI{200}{\nano\meter}$ STM Bild \ref{fig:res/hopg_131} aufgenommen und eine geeignete Flache Stelle gesucht.
\jafps{res/hopg_131}{Übersichtsaufnahme der Graphitoberfläche im konstanten Strom Modus}{0.45}



\section{Auswertung}

\jafps{res/hopg_166}{hopg go brr}{0.45}
