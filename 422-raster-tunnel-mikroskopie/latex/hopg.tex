\chapter{Highly Ordered Pyrolitic Graphite (HOPG)}

Highly Ordered Pyrolitic Graphite (HOPG) ist ein Kristall, bestehend aus Schichten von jeweils hexagonal angeordneten Kohlenstoffatomen. Die Schichten sind zueinander verschoben, sodass jeweils die Hälfte der Kohlenstoffatome über solchen aus der nächsten Lage liegen und die andere Hälfte über entsprechenden Hohlstellen \cite{graphite}. Dies ist skizziert in Abbildung \ref{fig:graphit_gitter}.

\jafps{graphit_gitter}{Schematische Darstellung der Kristallstruktur von HOPG\cite{hopg_sketch}}{0.4}

Die Schichten zueinander sind nur schwach über Van-der-Vaals Kräfte gebunden, innerhalb der Brücken sind die Kohlenstoffatome kovalent, also deutlich stärker gebunden \cite{graphite}. Dies wird beim Vorbereiten der Probe ausgenutzt, da es erlaubt, ganze Schichten relativ leicht abzulösen während diese in sich jeweils intakt bleiben.

Die Verschiebung der Schichten zueinander führt dazu, dass jedes zweite Elektron an die darunterliegende Schicht gebunden ist, während die übrigen keine Bindung zu dieser haben. Dies führt durch nach unten verschobene Elektronenorbitale, der an die untere Schicht gebundenen Elektronen, dafür, dass diese Elektronen weit weniger über den Tunneleffekt im RTM messbar sind. Infolgedessen wird mit dem RTM nur jedes zweite Elektron gemessen. Diese haben dann in der Aufnahme einen Abstand von $d = \SI{0.246}{\nano\meter}$ und lassen sich so als gleichseitige Dreiecke beobachten \cite[13]{Versuchsanleitung422}.


\section{Durchführung}
\label{sec:hopg_messung}
Die HOPG Probe wird zunächst unter dem USB Mikroskop dokumentiert (siehe Abbildung \ref{fig:hopg_scaled} und Skalierungsnetzchen in Abbildung \ref{fig:netz_hopg_scaled.pdf}).

\jafps{hopg_scaled}{USB-Mikroskop Aufnahme der genutzten HOPG-Probe.}{0.4}

Anschließend wird die Probe vorbereitet indem ein Stück Klebeband-Film auf die Probe geklebt und dann abgezogen wurde. Dies sorgt dafür, dass eine frische, unverschmutzte und näherungsweise ebene Graphitschicht freigelegt wird.
Es wird zunächst im CCM ein $\SI{200}{\nano\meter} \times \SI{200}{\nano\meter}$ RTM-Bild \ref{fig:res/hopg_131} aufgenommen und eine geeignete flache Stelle gesucht.
\jafps{res/hopg_131}{Übersichtsaufnahme der Graphitoberfläche im konstanten Strom Modus ($\SI{200}{\nano\meter}$, z-Darstellung), $v_R = \SI{1000.0}{\nano\meter\per\second}$, $I_\text{Setpoint} = \SI{1}{\nano\ampere}$, $P=1000$, $I=2000$}{0.4}


In der unteren linken Hälfte der Aufnahme \ref{fig:res/hopg_131} findet sich eine Stufe, ansonsten wird die Oberfläche als eben identifiziert. Es wird im weiteren ein Ausschnitt in der rechten oberen Hälfte von Abbildung \ref{fig:res/hopg_131} genutzt. Schrittweise wird der Ausschnitt verkleinert und im CCM die Glätte des Ausschnitts verifiziert. Bei einer Größe von $\SI{4}{\nano\meter} \times \SI{4}{\nano\meter}$ wird anschließend auf den konstanten Höhen-Modus (CHM) gewechselt, in dem die Konstante P auf 0 und I auf 4 gesetzt werden.

Nun werden Aufnahmen mit atomarer Auflösung hergestellt (siehe \ref{fig:res/hopg_166_with_lines}).

\section{Auswertung}


In den Aufnahmen erwarten wir entsprechend Abbildung \ref{fig:graphit_gitter}, ein Muster gleichseitiger Dreiecke mit Seitenlänge $\SI{0.246}{\nano\meter}$ \cite{graphite}. Dieses Muster ist in den Aufnahmen (\ref{fig:hopg_166_with_lines}, \ref{fig:res/hopg_165}) optisch zu erkennen, die Gleichseitigkeit der Dreiecke wird im folgenden überprüft.

Eine gelungene Aufnahme wird nun genutzt um dort den Abstand mehrerer Maxima zueinander zu vermessen. Hierzu wurde die Mess-Funktionalität von gwyddion \cite{gwyddion} verwendet.

% \jafps{res/hopg_166}{\todo{scanspeed}}{0.4}
\jafps{res/hopg_166_with_lines}{Messung des HOPG mit Messlinien zur Bestimmung des Kristallgitters. Bildgröße \SI{4.04}{\nano\meter}, Strombild, $v_R = \SI{157.20}{\nano\meter\per\second}$, $I_\text{Setpoint} = \SI{751}{\nano\ampere}$, $P=0$, $I=4$}{0.4}


Zum besseren Mitteln wurde jeweils der Abstand von fünf Maxima bestimmt. Außerdem werden die Winkel der Linien in denen die Maxima liegen vermessen.

\begin{table}
	\centering
	\begin{tabular}{|c|c|c|}
		\hline
		Index & $\varphi / \unit{\degree}$ & Länge $L / \unit{\nano\meter}$ \\
		\hline
		1     & $\num{161(1)}$             & $\num{1.24(3)}$                \\
		2     & $\num{162(1)}$             & $\num{1.26(3)}$                \\
		6     & $\num{160(1)}$             & $\num{1.26(3)}$                \\
		\hline
		3     & $\num{40(1)}$              & $\num{1.26(3)}$                \\
		4     & $\num{40(1)}$              & $\num{1.26(3)}$                \\
		5     & $\num{39(1)}$              & $\num{1.29(3)}$                \\
		\hline
	\end{tabular}
	\caption{Längen und Winkel relativ zur horizontalen der in \ref{fig:res/hopg_166_with_lines} markierten Linien. Die Unsicherheiten ergeben sich aus der Ausdehnung der Maxima.}
	\label{tab:hopg_measure}
\end{table}

Die Werte in Tabelle \ref{tab:hopg_measure} lassen sich in zwei Gruppen aufspalten, die den zwei Richtungen entsprechen, in denen die Kohlenstoffatome in Reihen liegen. Für jede Richtung wird jetzt der Mittelwert des Abstands zwischen je zwei Maxima berechnet, die Unsicherheit wird mittels Gaußscher Fehlerfortpflanzung berechnet\cite[8-9]{gerthsen}. Es ergibt sich für die erste Richtung (repräsentiert durch die Linien 1, 2, 6) ein mittlerer Winkel von $\varphi_1 = \SI{161(1)}{\degree}$ und ein Abstand von $L_1 = \SI{0.251(4)}{\nano\meter}$. Die selben Berechnungen für die zweite Gruppe (aus Linie 3, 4 und 5) ergeben $\varphi_2 = \SI{40(1)}{\degree}$ und $L_2 = \SI{0.254(4)}{\pico\meter}$.
Die Gitterabstände $L_1$ und $L_2$ sind zueinander konsisten und liegen innerhalb einer Standardabweichung zueinander. Der Erwartete Gitterabstand $d=\SI{0.246}{\nano\meter}$ liegt in einer $2\sigma$-Umgebung zu dem Messwert. Der Winkelunterschied beträgt $\Delta\varphi = \SI{121(1)}{\degree}$, was dem erwarteten Außenwinkel eines gleichseitigen Dreiecks ($\SI{120}{\degree}$) gut entspricht. Somit konnte sowohl die Winkel als auch Abstände des HOPG Kristallgitters gemessen werden, wenngleich bei den Gitterabständen nur eine übereinstimmung innerhalb einer $2\sigma$-Umgebung zu vorliegt.
Visuell ließ sich die elektronische Struktur von HOPG gut darstellen. Insgesamt lässt dies auf ein recht gut kalibriertes RTM schließen. Die leichten Abweichungen bei den gemessenen Längen lassen sich nicht eindeutig zuordnen, vorstellbar sind aber Hystereseeffekte der Piezoelemente oder Temperaturunterschiede zur Temperatur bei der Kalibration und dadurch veränderte elektrische oder mechanische Eigenschaften des Regelkreises oder der Piezoelemente.
