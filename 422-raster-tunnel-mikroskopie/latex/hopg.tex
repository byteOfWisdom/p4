\chapter{Highly Ordered Pyrolitic Graphite (HOPG)}

Highly ordered Pyrolitic Graphite (HOPG) ist ein Kristall, bestehend aus Schichten von jeweils hexagonal angeordneten Kohlenstoffatomen. Die Schichten sind zueinander verschoben, so dass jeweils die Hälfte der Kohlenstoffatome über solchen aus der nächsten Lage liegen und die andere Hälfte über entsprechenden Hohlstellen. Dies ist skizziert in Abbildung \todo{abbildung} \todo{cite}.

Die Schichten zueinander sind nur schwach über Van-der-Vaals Kräfte gebunden, innerhalb der Brücken sind die Kohlenstoffatome kovalent, also deutlich stärker gebunden \todo{cite}. Dies wird beim Vorbereiten der Probe ausgenutzt, da es erlaubt, ganze schichten relativ leicht abzulösen während diese in sich in Takt bleiben.

Die Verschiebung der Schichten zueinander führt dazu, dass jedes zweite Elektron an die darunterliegende Schicht gebunden ist, während die übrigen keine Bindung zu dieser haben. Dies führt durch verschobene Elektronenorbitale, der an die untere Schicht gebundenen Elektronen, dafür, dass diese Elektronen weit weniger Messbar sind. In Folge dessen wird mit dem STM nur jedes zweite Elektron gemessen. Diese haben dann theoretisch einen Abstand von $d = \SI{0.246}{\nano\meter}$ und bilden gleichseitige Dreiecke \cite{Versuchsanleitung422}.


\section{Durchführung}
Die HOPG Probe wird zunächst unter dem USB Mikroskop dokumentiert (siehe \ref{fig:HOPG_optisch})
\todo{bild mit skala}

Anschließend wird die Probe vorbereitet in dem ein Stück Tesa Film auf die Probe geklebt und dann abgezogen wurde. Dies sorgt dafür, dass eine Frische, unverschmutzte Graphitschicht freigelegt wird.
Es wird zunächst im CCM ein $\SI{200}{\nano\meter} \times \SI{200}{\nano\meter}$ STM Bild \ref{fig:res/hopg_131} aufgenommen und eine geeignete Flache Stelle gesucht.
\jafps{res/hopg_131}{Übersichtsaufnahme der Graphitoberfläche im konstanten Strom Modus \todo{scanspeed}}{0.4}

In der unteren linken Hälfte der Aufnahme \ref{fig:res/hopg_131} fand sich eine Stufe, ansonsten wird die Oberfläche als eben identifiziert. Es wird im weiteren ein Ausschnitt in der rechten oberen Hälfte von Abbildung \ref{fig:res/hopg_131} genutzt. Schrittweise wird der Ausschnitt verkleinert und im CCM die Glätte des Ausschnitts verifiziert. Bei einer Größe von $\SI{4}{\nano\meter} \times \SI{4}{\nano\meter}$ wird anschließend auf den konstanten Höhen Modus (CHM) gewechselt, in dem die Konstante P auf 0 und I auf 4 gesetzt werden.

Nun werden Aufnahmen mit atomarer Auflösung hergestellt (siehe \ref{fig:res/hopg_166}).

\section{Auswertung}

In den Aufnahmen erwarten wir gemäß \todo{refrence graphic}, ein Muster gleichseitigen Dreiecken mit Seitenlänge $\SI{0.246}{\nano\meter}$. Eine gelungene Aufnahme wird nun genutzt um dort den Abstand mehrerer Maxima zueinander zu vermessen.

% \jafps{res/hopg_166}{\todo{scanspeed}}{0.4}
\jafps{res/hopg_166_with_lines}{\todo{scanspeed}}{0.4}


Zum besseren Mitteln wurde jeweils der Abstand von fünf Maxima bestimmt. Außerdem werden die Winkel der Linien in denen die Maxima liegen vermessen.

\begin{table}
\centering
\begin{tabular}{|c|c|c|}
\hline
Index & $\varphi / \unit{\degree}$ & Länge $L / \unit{\pico\meter}$ \\
\hline
1 & $\num{161.1(5)}$ & $\num{1236.0(5)}$ \\
2 & $\num{161.8(5)}$ & $\num{1261.7(5)}$ \\
3 & $\num{39.5(5)}$ & $\num{1256.8(5)}$ \\
4 & $\num{40.2(5)}$ & $\num{1260.6(5)}$ \\
5 & $\num{39.4(5)}$ & $\num{1293.7(5)}$ \\
6 & $\num{160.3(5)}$ & $\num{1257.3(5)}$ \\
\hline
\end{tabular}
\caption{Längen und Winkel relativ zur horizontalen der in \ref{fig:res/hopg_166_with_lines} markierten Linien.}
\label{tab:hopg_measure}
\end{table}
