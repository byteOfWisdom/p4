\chapter{Einleitung}

\section{Versuchsziel}

Rastertunnelmikroskopie ist eine wichtige Methode, um Strukturen zu untersuchen, die mit klassischen optischen Mikroskopen nicht aufgelöst werden können. Auch eine atomare Skala der Auflösung ist dabei erreichbar \cite[2]{Versuchsanleitung422}.
In dem Versuch wird ein Rastertunnelmikroskop (RTM) genutzt, um die Oberflächentopologie und damit die Kristallstruktur von auf Saphir oder Silizium aufgedampften Goldkügelchen und anschließend von hochgeordneten pyrolytischem Graphit (HOPG) zu untersuchen und letzteres auch auf atomarer Skala aufzulösen \cite[1]{Versuchsanleitung422}. Außerdem wird auch die Oberflächenstruktur von Molybdändisulfid (\moszwei) mit dem RTM untersucht.

\section{Funktionsweise eines RTM}

Rastertunnelmikroskopie nutzt das Phänomen des messbaren, ortsabhängigen Elektronenstroms (Tunnelstrom)zwischen einer elektrisch leitfähigen Oberfläche einer Probe und einer feiner Metallspitze, wenn diese in einem kleinen Abstand $z$ über die Oberfläche bewegt wird. Dabei wird die Bewegung der Spitze über die Probe als fein über Piezokristalle gesteuerte Rasterbewegung in der x-y-Ebene vorgenommen und die Höhe $z>0$ und der Tunnelstrom $I$ für jeden Rasterpunkt (Pixel des entstehenden Bildes) gemessen. So können Strukturen der Probenoberfläche untersucht werden.

\subsection{Tunnelstrom}
Bei klassischer Betrachtung wird überhaupt kein Tunnelstrom erwartet, da Elektronen aus leitfähigen Materialien, mit Energien unterhalb des Betrags einer Potentialbarriere $E < V$ sich niemals innerhalb oder hinter der Barriere aufhalten können. In der quantenmechanischen Betrachtung von Elektronen als Wellenfunktion gibt es jedoch eine nichtverschwindene Wahrscheinlichkeit, dass sich ein Elektron auch im Bereich der Barriere oder dahinter aufhält.

%Grundsätzlich nutzt ein RTM diesen quantenmechanischen Tunneleffekt, sowie feine Positionierung über Piezokristalle, aus um die Oberfläche einer Probe mit einer atomar dünnen Spitze abzurastern und Daten über ihre Struktur zu messen.
% \subsection{Tunnelstrom}
% Bei quantenmechanischer Betrachtung können Teilchen durch Potentialbarrieren Tunneln, die bei klassischer Betrachtung das Teilchen stoppen sollten. \todo{cite this}

Der resultierende Tunnelstrom ist hängt exponentiell vom Abstand ab und ist sehr empfindlich bei Änderungen in den Größenordnungen weniger Angström. Es gilt die folgende Proportionalität \cite{binnigSTM}:

\begin{align}
	I \propto U_T \exp(-A \Phi^{1/2} z)
\end{align}



Wobei


\subsection{Piezoeffekt}
Der Piezoeffekt beschreibt den Zusammenhang zwischen elektrischer Spannung und Verformung von Materialien mit polaren Achsen.
Dieser Zusammenhang ist näherungsweise linear;

\begin{align}
	\Delta x = \delta U
\end{align}

Dieser Effekt ist in der Größenordnung von $10^{-11} - 10^{-9} \unit{\meter\per\volt}$ und erlaubt somit relativ einfach Positionsregelung auf atomaren Größenskalen. \cite[339-341]{gerthsen}


\subsection{Regelkreis}
Der Abstand der Spitze über der Probe muss geregelt werden, damit der Abstand stets klein genug ist, damit ein messbarer Tunnelstrom fließt ohne jedoch die Probe zu berühren und so potentiell die Spitze zu zerstören.

Zu diesem Zweck wird eine PID-Schleife verwendet. Dies ist eine in der Regelungstechnik übliche Art, ein System zu regeln, so dass es sich möglichst effizient einem Sollpunkt annährt. \todo{cite}

\todo{macht es Sinn hier die Standardform der PID Regelgleichung hinzupacken und zu erläutern??}

Wird der Proportionalfakor auf Null gesetzt und die Integrationszeit sehr lang gewählt, folgt die Spitze nur noch der generellen Tendenz oder Schieflage der Oberfläche, nicht mehr aber Unebenheiten. Dies entspricht dem Wechsel vom konstanten Strom Modus (CCM) zum konstanten Höhen Modus (CHM). \cite{Versuchsanleitung422}
