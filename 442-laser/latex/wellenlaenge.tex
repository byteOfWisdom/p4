\chapter{Messung der Wellenlänge des Lasers}

\section{Theoretische Überlegungen} \label{sec:theo_wellenlaenge}

Um die tatsächliche Wellenlänge des Laserlichts zu bestimmen, eignet sich ein Gitter. Dabei wird der Laserstrahl genau parallel zur Gitternormalen
auf ein Reflexions- oder Transmissionsgitter mit Spaltabstand d.h. Gitterkonstante $g$ eingestrahlt (Fall normalen Einfalls). Aus den Maxima des resultierenden Interferezmusters auf
einem Schirm kann mithilfe der Bedingung für den Gangunterschied für konstruktive Interferenz $\delta = n \lambda$ mit $\lambda$ als Wellenlänge des Lasers die Gittergleichung aufgestellt werden \cite[308]{Demtroeder2}:
\begin{align}
	\label{eq:gitter}
	m \lambda = g \cdot (sin(\alpha)+sin(\beta_m))
\end{align}
Sie ist in Abbildung \ref{fig:gitter.png} grafisch dargestellt. Hier wurde als Einfallswinkel normaler Einfall gewählt, also $\alpha = \SI{0}{\degree}$.

\jafps{gitter.png}{Illustration des Gangunterschieds und Ein- und Ausfallswinkel eines Lichtstrahls an einem optischen Gitter. Abbildung entnommen aus \cite{LDbalmerserie}.}{0.3}

Das Maximum der $n=0$-ten Ordnung wird nach Gl. \ref{eq:gitter} wegen $0\cdot \lambda = g\sin(\beta_0)=0$, also $\beta_0=\SI{0}{\degree}$, nicht abgelenkt.
Dies ist der Vorteil eines Transmissions- gegenüber einem Reflexionsgitter,
bei welchen aufgrund eines notwendigen Einfallswinkels von $\alpha > 0$ ein Ablenkwinkel auch für das 0-te Maximum des Interferenzmusters zu beachten wäre.
Aus dem Abstand des Schirms $f$ und dem Abstand $d_m$ der m-ten Maximaordnungen von der Projektion der Gitternormale auf den Schirm bzw.
hier der Position des 0-ten Maximums kann der Ablenkwinkel
der Maxima bestimmt werden:
\[\tan(\beta_m)=\frac{d_m}{f} \ergo \beta_m = \arctan(\frac{d_m}{f})\]

So lässt sich dann die Wellenlänge aus der Messung der Abstände $d_m$ der verschiedenen sichtbaren Maxima bestimmen,
indem der Term $g \cdot \sin \left( \beta_m \right)$ gegen die Ordnung $m$ aufgetragen wird:
\begin{align}
	\label{eq:lambda_gitter}
	g\cdot \sin\left(\arctan\left(\frac{d_m}{f}\right)\right) = m \lambda
\end{align}

Die Steigung dieser Gerade entspricht dann eben der Wellenlänge $\lambda$.

\section{Durchführung \& Messung}

Um die Wellenlänge zu bestimmen, wurde im Experiment ein Transmissionsgitter direkt vor den ebenen Resonatorspiegel auf die Schiebe eingesetzt.
Dann wurde der Abstand zu einem Schirm gemessen sowie der Abstand der auf dem Schirm sichtbaren Maxima zum nicht-abgelenkten Maxima der 0.-ten Ordnung.
Zum Messung der Abstände wurde ein Zollstock genutzt, aus dessen Skalenunterteilung der Fehler der Abstandsmessungen
als $\Delta x=\SI{5}{\milli\meter}$ abgeschätzt wurde.
Messunsicherheiten werden stets (sofern nicht explizit anders erwähnt) über gaußsche Fehlerfortpflanzung propagiert. \todo{cite}
Zur Reduktion von statistischen Fehlern durch eine größere Anzahl an Messwerten wurde die Messung für zwei verschiedene Gitter
mit verschiedenen Gitterkonstanten ($500$ und $1000$ Striche pro $\unit{mm}$)
jeweils dreimal durchgeführt.
Die gemessenen Werte sind in Tabellen \ref{tab:gitter_500} und \ref{tab:gitter_1000} aufgeführt.

\begin{table}[H]
	\centering
	\begin{tabular}{|c|c|c|}
		\hline
		Messung & Ordnung $m$ & $d_m / \unit{m}$   \\ \hline
		1       & 1           & \num{15(0.5)e-2}   \\
		1       & 2           & \num{11(0.5)e-2}   \\
		1       & 3           & \num{5(0.5)e-2}    \\
		2       & 1           & \num{15(0.5)e-2}   \\
		2       & 2           & \num{10.9(0.5)e-2} \\
		2       & 3           & \num{4.8(0.5)e-2}  \\
		3       & 1           & \num{15(0.5)e-2}   \\
		3       & 2           & \num{10.7(0.5)e-2} \\
		3       & 3           & \num{4.5(0.5)e-2}  \\
		\hline
	\end{tabular}
	\caption{Messung der Ablenkdistanzen $d_m$ für das Gitter mit $500$ Strichen pro mm. Der Abstand zum Schirm wurde gemessen als $f=\SI{13(0.5)e-2}{\meter}$.}
	\label{tab:gitter_500}
\end{table}

\begin{table}[H]
	\centering
	\begin{tabular}{|c|c|c|} \hline
		Messung & Ordnung $m$ & $d_m / \unit{m}$   \\ \hline
		1       & 1           & \num{15(0.5)e-2}   \\
		1       & 2           & \num{14.5(0.5)e-2} \\
		1       & 3           & \num{5.5(0.5)e-2}  \\
		2       & 1           & \num{15(0.5)e-2}   \\
		2       & 2           & \num{14.4(0.5)e-2} \\
		2       & 3           & \num{5.3(0.5)e-2}  \\
		3       & 1           & \num{15(0.5)e-2}   \\
		3       & 2           & \num{14.5(0.5)e-2} \\
		3       & 3           & \num{5.2(0.5)e-2}  \\
		\hline
	\end{tabular}
	\caption{Messung der Ablenkdistanten $d_m$ für das Gitter mit 1000 Strichen pro mm. Der Abstand zum Schirm wurde gemessen als $f=\SI{14(0.5)e-2}{\meter}$.}
	\label{tab:gitter_1000}
\end{table}

\section{Auswertung}

Für beide Gitter wurden wie in Abschnitt \ref{sec:theo_wellenlaenge} die Terme $g\cdot \sin\left(\arctan\left(\frac{d_m}{f}\right)\right)$
gegen die (vermutete) Ordnung $m$ aufgetragen.
Dabei wurden die Messreihen der beiden Gitter getrennt aufgetragen, dies ist in den Abbildungen \ref{fig:gitter500.pdf} und \ref{fig:gitter1000.pdf} dargestellt.
Die beiden so bestimmten Werte für $\lambda$ des Lasers sind: $\lambda_{500}=\SI{616(5)}{\nano\meter}$ und $\lambda_{1000}=\SI{275(51)}{\nano\meter}$.

\jafps{gitter500.pdf}{Auftragung von $g\cdot \sin\left(\arctan\left(\frac{d_m}{f}\right)\right)$ gegen die Ordnung $m$ für das Gitter mit $500$ Strichen pro mm. Das Bestimmtheitsmaß für die Güte der Anpassung $R^2=0,999$ ($0\leq R^1\leq 1$) zeigt, dass die Geradenpassung gut zu den Messwerten passt.}{0.6}
\jafps{gitter1000.pdf}{Auftragung von $g\cdot \sin\left(\arctan\left(\frac{d_m}{f}\right)\right)$ gegen die Ordnung $m$ für das Gitter mit $1000$ Strichen pro mm. Aus dem Wert der Anpassungsgüte $R^2=0,626$ wird deutlich, dass die Geradenanpassung hier nicht zu den gemessenen Werten passt.}{0.6}

Es fällt sofort auf, dass sich aus der Messreihe mit dem Gitter mit $1000$ Strichen pro mm als Steigung eine Wellenlänge ergibt, die sehr weit von der
Wellenlänge des angestrebten Laserübergangs von $\lambda_\text{Lit}=\SI{632.6}{\nano\meter}$ entfernt ist (auch deutlich weiter entfernt als die zu erwartende Linienbreite des Übergangs,
siehe hierzu Abschnitt \ref{sec:theorielaser}). Auch eine der anderen möglichen Laserübergänge vom angeregten Zustand der Ne-Atome
(Abschnitt \ref{sec:theorielaser}) entspricht diese Wellenlänge nicht.
Das Laserlicht wurde außerdem auch optisch als deutliches rot im optischen Bereich verortet. Es ist also damit zu rechnen, dass die Wellenlänge sich im optischen roten Bereich befindet.
Im Gegensatz dazu entspricht die Wellenlänge aus der Messreihe des anderen Gitters mit $500$ Strichen pro mm,$\lambda_{500}=\SI{616(5)}{\nano\meter}$, deutlich besser der zu erwartenden Wellenlänge.
Es ist also damit sehr wahrscheinlich, dass die Messung des Gitters mit $1000$ Strichen pro mm systematische Fehler enthält und wird daher im folgenden als unplausibel verworfen.
Es ist zu erwähnen, dass die Auswertung der Geradenanpassung auch keinen plausibleren Wellenlängenwert liefert, wenn angenommen wird, dass die Schätzung der Ordnungen,
die beobachtet und vermessen werden konnten, fehlerhaft war, und entsprechend systematisch alternative Ordnungszuordnungen durchprobiert wurden.
Analog lieferte auch die Exklusion der drei Messungen für jeweils eine der drei beobachteten Ordnungen keine sinnvolleren Werte.
Als mögliche Fehlerquelle dieser drei Messreihen ist denkbar, dass der Zollstock, der zur Messung der Abstände der Maxima genutzt wurde,
nicht plan zur Schirmebene bzw. zum Gitter positioniert wurde, sondern zur Seite gekippt wurde. Daher ist auch kein konstanter Versatz der gemessenen Abstände zu
einem zu erwarteten Wert möglich auf die Werte zu rechnen, sondern die Verzerrung der Abstände zur aufgezeichneten Skala mit zunehmenden Abstand zum 0-ten Maximum wächst nonlinear.

Durch die mehrfache Messung des Interferenzmusters mit dem Gitter mit $500$ Strichen pro mm ist trotzdem noch eine ausreichende Menge Datenpunkte vorhanden, um eine sinnvolle Auswertung durchführen zu können.
Der geringe Unterschied von $\lambda_{500}$ zum Literaturwert $\lambda_\text{Lit}$ von $2,8\%$ (bei Beachtung der $1\sigma$-Umgebung: $1,9\%-3,6\%$) zeigt dies.
In Abschnitt \ref{sec:optical_analyzer} wird gezeigt werden, dass in der Laserkavität mehrere longitudinale und vermutlich auch transversale Moden anschwingen,
deren Frequenzen sich jeweils um (aus der Messung mit dem Spektrumanalysator) $\SI{293.3(0.5)}{\mega\hertz}$ unterscheiden, also auch die Wellenlängen der schwingenden Moden unterschiedlich sind. Der HeNe-Laser im genutzten Aufbau hat insgesamt so wenig Überlapp zwischen den Resonatormoden des Laserkavität und keine frequenzselektiven Bauteile und ist daher in dieser Betriebsart ein sogenannter \textit{multi-mode} Laser, es schwingen daher hier verschiedene Lasermoden.
Es liegt nahe, dass hier bei der Wellenlängenmessung der Effekt der sich überlagernden Moden des Resonators sichtbar wird, der die messbare Wellenlänge des am ebenen Resonatorspiegel ausgekoppelten Laserlichts
im Vergleich zur Referenzwellenlänge des Laserübergangs von $3s\to 2p$ verschiebt.
