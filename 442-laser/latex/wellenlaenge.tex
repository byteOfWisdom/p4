\chapter{Messung der Wellenlänge des Lasers}

\section{Theoretische Überlegungen}

Um die tatsächliche Wellenlänge des Laserlichts zu bestimmen, eignet sich ein Transmissionsgitter. Dabei wird der Laserstrahl genau parallel zur Gitternormalen
auf ein Reflexionsgitter mit Spaltabstand d.h. Gitterkonstante $g$ eingestrahlt (Fall normalen Einfalls). Aus den Maxima des resultierenden Interferezmusters auf
einem Schirm kann mithilfe der Bedingung für den Gangunterschied für konstruktive Interferenz $\delta = n \lambda$ mit $\lambda$ als Wellenlänge des Lasers die Gittergleichung aufgestellt werden:
\begin{align}
	\label{eq:gitter}
	m \lambda = g \cdot (sin(\alpha)+sin(\beta_m))
\end{align}
Sie ist in Abbildung \ref{fig:gitter.png} grafisch dargestellt. Hier wurde als Einfallswinkel normaler Einfall gewählt, also $\alpha = \SI{0}{\degree}$.

\jafps{gitter.png}{Illustration des Gangunterschieds und Ein- und Ausfallswinkel eines Lichtstrahls an einem optischen Gitter. Abbildung entnommen aus \cite{LDbalmerserie}.}{0.3}

Das Maximum der $n=0$-ten Ordnung wird nach Gl. \ref{eq:gitter} wegen $0\cdot \lambda = g\sin(\beta_0)=0$, also $\beta_0=\SI{0}{\degree}$, nicht abgelenkt.
Dies ist der Vorteil eines Transmissions- gegenüber einem Reflexionsgitter,
bei welchen aufgrund eines notwendigen Einfallswinkels von $\alpha > 0$ ein Ablenkwinkel auch für das 0.-te Maximum des Interferenzmusters zu beachten wäre.
Aus dem Abstand des Schirms $f$ und dem Abstand $d_m$ der m-ten Maximaordnungen von der Projektion der Gitternormale auf den Schirm bzw.
hier der Position des 0-ten Maximums kann der Ablenkwinkel
der Maxima bestimmt werden:
\[\tan(\beta_m)=\frac{d_m}{f} \ergo \beta_m = \arctan(\frac{d_m}{f})\]

So lässt sich dann die Wellenlänge aus der Messung der Abstände $d_m$ der verschiedenen sichtbaren Maxima bestimmen,
indem der Term $g \cdot \sin \left( \beta_m \right)$ gegen die Ordnung $m$ aufgetragen wird:
\begin{align}
	\label{eq:lambda_gitter}
	g\cdot \sin\left(\arctan\left(\frac{d_m}{f}\right)\right) = m \lambda
\end{align}


Die Steigung dieser Gerade entspricht dann eben der Wellenlänge $\lambda$.

\section{Durchführung \& Messung}

Um die Wellenlänge zu bestimmen, wurde im Experiment ein Transmissionsgitter direkt vor den ebenen Resonatorspiegel auf die Schiebe eingesetzt.
Dann wurde der Abstand zu einem Schirm gemessen sowie der Abstand der auf dem Schirm sichtbaren Maxima zum nicht-abgelenkten Maxima der 0.-ten Ordnung.
Zum Messung der Abstände wurde ein Zollstock genutzt, aus dessen Skalenunterteilung der Fehler der Abstandsmessungen
als $\Delta x=\SI{5}{\milli\meter}$ abgeschätzt wurde.
Zur Reduktion von statistischen Fehler durch eine größere Anzahl an Messwerten wurde die Messung für zwei verschiedene Gitter
mit verschiedenen Gitterkonstanten ($500$ und $1000$ Striche pro $\unit{mm}$)
jeweils dreimal durchgeführt.
Die gemessenen Werte sind in Tabellen \ref{tab:gitter_500} und \ref{tab:gitter_1000} aufgeführt.

\begin{table}[H]
	\centering
	\begin{tabular}{|c|c|c|}
		\hline
		Messung & Ordnung $m$ & $d_m / \unit{m}$   \\ \hline
		1       & 1           & \num{15(0.5)e-2}   \\
		1       & 2           & \num{11(0.5)e-2}   \\
		1       & 3           & \num{5(0.5)e-2}    \\
		2       & 1           & \num{15(0.5)e-2}   \\
		2       & 2           & \num{10.9(0.5)e-2} \\
		2       & 3           & \num{4.8(0.5)e-2}  \\
		3       & 1           & \num{15(0.5)e-2}   \\
		3       & 2           & \num{10.7(0.5)e-2} \\
		3       & 3           & \num{4.5(0.5)e-2}  \\
		\hline
	\end{tabular}
	\caption{Messung der Ablenkdistanzen $d_m$ für das Gitter mit $500$ Strichen pro mm. Der Abstand zum Schirm wurde gemessen als $f=\SI{13(0.5)e-2}{\meter}$.}
	\label{tab:gitter_500}
\end{table}

\begin{table}[H]
	\centering
	\begin{tabular}{|c|c|c|} \hline
		Messung & Ordnung $m$ & $d_m / \unit{m}$   \\ \hline
		1       & 1           & \num{15(0.5)e-2}   \\
		1       & 2           & \num{14.5(0.5)e-2} \\
		1       & 3           & \num{5.5(0.5)e-2}  \\
		2       & 1           & \num{15(0.5)e-2}   \\
		2       & 2           & \num{14.4(0.5)e-2} \\
		2       & 3           & \num{5.3(0.5)e-2}  \\
		3       & 1           & \num{15(0.5)e-2}   \\
		3       & 2           & \num{14.5(0.5)e-2} \\
		3       & 3           & \num{5.2(0.5)e-2}  \\
		\hline
	\end{tabular}
	\caption{Messung der Ablenkdistanten $d_m$ für das Gitter mit 1000 Strichen pro mm. Der Abstand zum Schirm wurde gemessen als $f=\SI{14(0.5)e-2}{\meter}$.}
	\label{tab:gitter_1000}
\end{table}

\section{Auswertung}
