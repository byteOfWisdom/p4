\chapter{Aufbau des Lasers}

\section{Theoretische Überlegungen}

% Was ist ein Laser

Der Laser, der im folgenden aufgebaut und untersucht werden soll, ist ein HeNe-Laser mit dem folgenden Termschema.

\jafps{termschema.png}{Termschema des aufzubauenden HeNe-Lasers mit Laserwellenlänge $\lambda=\SI{632.8}{\nano\meter}$. Grafik entnommen aus \cite[16]{versuchsanleitung442}}{0.8}

Das aktive Medium dieses Lasers ist ein Helium-Neon-Gasgemisch (Verhältnis etwa 10:1). Durch das Anschließen einer externen Hochspannungsquelle werden Heliumatome ionisiert, wobei
die freien Elektronen beschleunigt werden und dann mit weiteren Heliumatomen stoßen. Dadurch werden diese aus dem Grundzustand in den Zustand $2^1S_0$ angeregt. Die angeregten He-Atome stoßen
mit Neon-Atome, die so vom Grundzustand in den $3s$-Zustand angeregt werden. Dieser Zustand ist metastabil, nach zufälliger spontaner Emission und Übergang in den niedrigenergetischeren $2p$-Zustand (zufällig Photon in Richtung der späteren Laserstrahlachse) werden
so weitere Ne-Atome im angeregten $3s$-Zustand zu stimulierter Emission gebracht. Das emittierte Photon bei jedem dieser Übergänge hat dabei die gewünschte Laserfrequenz von $\lambda=\SI{632.8}{\nano\meter}$. Da die vorhandenen Photonen
die Emission bei anderen $3s$-Ne-Atomen induzieren, sind die so emittierten Photonen bzw. die elektromagnetische Welle kohärent und (annäherernd, siehe unten zur Linienbreite) monochromatisch.
Die He-Ne-Gasröhre mit einem geringen Durchmesser wird zudem noch in einem Resonator aus einem planen und einem sphärischen Spiegel platziert, sodass eine Rückkoppplung des emittierten Lichts erfolgen kann.



%Grundsätzlich braucht jeder Laser eine Besetzungsinversion bezüglich der Besetzungsdichten
%von einem höheren und einem niedrigeren Energieniveau der Moleküle seines aktiven Mediums. Bei Abregung in den niedrigeren Zustan durch Emission eines Photons
