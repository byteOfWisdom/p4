\chapter{Fazit}

In diesem Versuch wurde ein HeNe-Laser für die Wellenlänge $\lambda=\SI{632.8}{\nano\meter}$ aufgebaut und der Laserstrahl hinsichtlich seiner Wellenlänge, seinem Polarisationszustand und des Modenspektrums des genutzten Laserresonators aus einem sphärischen und ebenen Spiegels untersucht, letzteres mithilfe eines externen Resonators. Die Justage des Lasers war sehr zeitaufwändig und verhinderte leider eine Messung des Laserstrahls für mehrere Laserresonatorlängen.

Die tatsächliche Wellenlänge des Laserlichts wurde mithilfe eines Transmissionsgitters und den Winkeln der ersten drei Maximaordnungen des beobachteten Interferenzmusters (Tabelle \ref{tab:gitter_500} auf $\lambda_{500}=\SI{616(5)}{\nano\meter}$ bestimmt. Die Abweichung zum Literaturwert (Übreinstimmung nur im Rahmen von einer $4\sigma$-Umgebung) ist vermutlich durch die mehreren longitudinalen und transversalen Moden mit zueinander verschobenen Wellenlängen, die im Resonator anschwingen und sich (ohne Modenkonkurrenz) überlagern, zu erklären.

Mithilfe eines Polarisators und einer Photodiode wurde anschließend der Polarisationzustand des aus dem Laserresonator ausgekoppelten Laserstrahls untersucht. Die ermittelten Photodiodenspannungen abhängig vom Polarisatorwinkel sind in Tabelle \ref{tab:polarisation} aufgeführt. Aus der Messung wurde der Polarisationsgrad als $PG = \num{0.970(0.002)}$ bestimmt und durch die Auftragung der Spannung $U$ als Maß der Lichtintensität gegen den Verdrehungswinkel der Einfluss des Polarisators auf den Laserstrahl die Gültigkeit des Gesetz' von Malus \eqref{eq:malus} bestätigt. Es konnte so festgestellt werden, dass der über Brewsterfenster linear polarisierte Laserstrahl tatsächlich näherungsweise linear polarisiert war.

Die Modenabstände des Lasers wurden auf zwei Weisen untersucht; zunächst wurde ein optischer Analysator mit veränderlicher Länge genutzt um das Verhältnis der Modenabstände der Laserkavität und des externen Resonators zu bestimmen. Hierbei wurden weit mehr Lasermoden gefunden als in das Verstärkungsprofil von Neon passen. Daraus wurde gefolgert, dass in der Laserkavität nicht nur Longitudalmoden sondern auch Transversalmoden geschwungen haben. Der gefundene Modenabstand der Lasermoden war $\Delta \nu = \SI{1.2(4)}{\mega\hertz}$. Dieser Abstand liegt in einer $1\sigma$ Umgebung des erwarteten Abstandes von $\Delta\nu_\text{theo} \approx \SI{290}{\mega\hertz}$. Dies unterstützt weiter die Annahme, dass auch Transversalmoden schwingen. Die spricht für eine relativ schlechte Justage des Laseroszillators.

Der Modenabstand wurde dann noch über einen Spektrumanalysator ermittelt. Hierbei wurde eine Differenzfrequenz von $\Delta\nu = \SI{293.3(5)}{\mega\hertz}$ gefunden. Diese ist konsistent mit dem Erwartungswert und - wenn Transversalmoden angeregt werden - mit dem des optischen Analysators.
