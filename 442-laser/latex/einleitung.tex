\chapter{Einleitung}

\section{Versuchsziel}

Laser (\textbf{L}ight \textbf{A}mplification by \textbf{S}timulated \textbf{E}mission of \textbf{R}adiation)
sind ein fundamentaler Bestandteil vieler moderner optischer Techniken.
Sie sind die Quelle kohärenten, (annähernd) monochromatischen Lichts im infraroten über optischen bis ultravioletten Spektralbereich
von elektromagnetischer Strahlung, welches hohe Strahlintensitäten bei einem sehr kleinen Durchmesser \cite[1-2]{Sigrist} erreicht.
Im durchgeführten Experiment wurde als sehr typische Form eines eines Gaslasers mit einem kontinuierlich betriebenen Strahl (\textit{continuous wave, CW}, im Gegensatz zu gepulsten Lasern)
der Helium-Neon(HeNe)-Laser aufgebaut und das resultierende Laserlicht auf seine Wellenlänge und den Polarisationszustand untersucht.
Außerdem werden die Moden des erzeugten Laserstrahls mithilfe eines externen Resonators und einem optischen Spektralanalysator untersucht.

\section{Funktionsweise von Lasern}
