\chapter{Modenabstand des Lasers}
\section{Optischer Analysator}

Es soll nun die Modenstruktur der Laserkavität mittels eines Spektrumanalysator untersucht werden.
Bei dem Analysator handelt es sich um einen externen, konfokalen Resonator, dessen Länge mittels eines Piezo-Kristalls verstellt wird. \cite{Versuchsanleitung442}

Der Modenabstand von TEM\footnote{Tranversalen elektrischen Moden} bei einem solchen Resonator ist gegeben als: \cite{Versuchsanleitung442}

\begin{align}
	\Delta \nu_{TEM} = \frac{c}{4 l} \label{eq:mode_spacing_ext}
\end{align}

Bei der Laserkavität hingegen wird davon ausgegangen, dass in erster Linie Longitudinalmoden angeregt werden mit einem Modenabstand von:

\begin{align}
	\Delta \nu = \frac{c}{2L}
\end{align}

Nach der Justage des Analysators erfolgt die Messung indem ein Dreiecksignal an das Piezoelement angelegt wird, dessen Spannung so lange variiert wird, bis auf dem Oszilliskop eine periodische Struktur zu erkennen ist. Hierbei ist relevant, dass die Modenstruktur sowohl des Analysators als auch des Lasers überlagert beobachtet wird.

Eine kurze Überschlagsrechnung zeigt auf, dass der Modenabstand des Lasers ca. eine Größenordnung kleiner als bei dem Resonator zu erwarten ist. Somit kann davon ausgegangen werden, dass die näher aneinanderliegenden, feineren Maxima Lasermoden entsprechen und die größeren, weiter auseinanderliegenden Maxima entsprechen Analysatormoden sind.

Die Länge der Laserkavität ließ sich gut Messen, mit $L = \SI{51.3(5)}{\centi\meter}$. Die Länge des Analysators hingegen ist schwer akkurat zu messen, da die genaue Position der Spiegel in den jeweiligen Bauteilen nicht eindeutig ist und die Bauteile durch ihre Größe und Form ein genaues Messen erschweren. Es ergibt sich eine Länge von $l = \SI{50(15)}{\milli\meter}$

In den aufgenommenen Oszillogrammen werden die Abstände der Analysatormoden und die Abstände der Lasermoden vermessen.

\jafpp{analysator_spectrum_0}{Oszillogramm des Spektrums}{analysator_spectrum_2}{zweite Messung des Selben}

Aufgrund des Linearen Spannungsverlaufs an dem Piezoelement und der dazu proportionalen Längenänderung werden alle Abstände an den Indizes der Datenpunkte berechnet.

Die lokalen Maxima der Kurven werden ermittelt. Der Abstand der Analysatormoden wird aus den Abständen der jeweils n-ten Maxima zueinander bestimmt. 

Die Abstände der nebeneinanderliegenden Moden wird berechnet und über lineare Regression der Achsenabschnitt bestimmt und dieser als mittlerer Abstand zweier Lasermoden gesehen. Bei dieser Regression kann eine Steigung nahe Null genutzt werden um Systemische Fehler auszuschließen.

Es ergibt sich eine Steigung von $a=\num{-2(3)}$ und ein Achenabschnitt von $b=\num{1226}{49}$. Die Steigung ist klein genug um in guter Näherung als Null betrachtet zu werden. Es liegt kein relevanter Trend in den Daten vor. 
Aus dem Mittelwert des  Analysatormodenabstands $\num{14724}$ wird das Verhältnis der Abstände bestimmt zu $q = 0.083(3)$.
Aus dem Modenabstand des Resonators und dem Verhältnis zum kleineren Modenabstand (des Lasers) kann nun der Modenabstand der Lasermoden ermittelt werden;

\begin{align}
	q &= \frac{\nu_\text{Laser}}{\nu_\text{TEM}}\\
	\imply \nu_\text{Laser} &= \nu_\text{TEM} \cdot q
\end{align}


\jafps{modespacing}{Modenabstände}{0.75}


Damit ergibt sich ein mittlerer Modenabstand der Lasermoden von $\Delta \nu_\text{Laser} = \SI{1.2(4)}{\mega\hertz}$.

Es ist hierbei wichtig, dass das Verstärkungsprofil von Neon nur eine Breite von ca. $\SI{1.5}{\giga\hertz}$ [CITE!!!] aufweist. Bei einem Transversalen Modenabstand von $\Delta\nu = \frac{c}{2 L} \approx \SI{290}{\mega\hertz}$ der Transversalmoden des Lasers können niemals mehr als 5 solcher Moden in dem Vestärkungsprofil liegen und somit emittieren.

Bei der Messung wurden hier allerdings 9 Lasermoden gefunden, wobei die erste vermutlich noch zu der vorherigen Periode gehört, was zu 8 tatsächlichen Lasermoden führt. Dies sind doppelt so viele wie maximal möglich sind, allerdings ist der Abstand der Moden erheblich kleiner als der Erwartungswert. Es muss insgesamt davon ausgeganen werden, dass in dem Laserresonator nicht nur Longitudinalmoden sondern auch Transversalmoden angeregt wurden. Dann wäre der Abstand zweier Longitudinalmoden der doppelt so groß wie der bestimmte. Damit ergibt sich ein Wert von $\Delta \nu_\text{Laser} = \SI{2.4(8)}{\mega\hertz}$. Dieser Wert liegt in einer 1-$\sigma$ Umgebung zum erwarteten Modenabstandes von Longitudinalmoden für unsere Resonatorlänge.

Die angeregten Transversalmoden sprechen für eine relativ schlechte Justage des Lasers.



\section{Spektrumanalysator}

Nun wird der Modenabstand der Lasermoden mit einem Spektrumanalysator bestimmt. Hierzu wird ausgenutzt, dass die Lasermoden sich überlagern. Diese Überlagerung kann dargestellt werden als Summe von Cosinusfunktionen mit verschiedenen Frequenzen. Anwenden der hinlänglich bekannten Additionstheoreme zeigt, dass das gesamtsignal mit der Summe der Einzelfrequenzen, der Differenz der Einzelfrequenzen und mit den jeweiligen Einzelfrequenzen oszilliert. \cite{Versuchsanleitung442}

Die Summe der Frequenzen, sowie die einzelnen Frequenzen sind im $\unit{\tera\hertz}$ Bereich und somit nicht als elektronische Signale übertragbar oder darstellbar [CITE?].

Somit verbleibt nur die Differenzfrequenz, welche dann genau der Differenz zwischen zwei Lasermoden entspricht. Diese wird mit einer schnellen Photodiode gemessen und auf einem Spektrumanalysator der Firma RIGOL, des Typs DSA 815 dargestellt.

Hierbei musste der dargestellte Bereich recht klein um den Erwartungswert gewählt werden, um überhaupt eine Messung durchführen zu können. Der Messbereicht wurde auf eine Spanne von $\SI{10}{\mega\hertz}$ eingestellt, mit einer Mittelfrequenz von $\SI{290}{\mega\hertz}$. So wurde ein schmales Maximum bei $\SI{293.3(5)}{\mega\hertz}$ gefunden, wobei es eine FWHM von etwa $\SI{1}{\mega\hertz}$ aufweist, woraus sich die Unsicherheit ergibt.

Dies deckt sich sowohl mit dem theoretisch erwarteten Modenabstand als auch der (weit weniger genauen) Messung mit dem optischen Analysator.

Leider wurde nicht daran gedacht auch bei $\SI{145}{\mega\hertz}$ nach einer Differenzfrequenz zu suchen um die Hypothese der angeregten Transversalmoden zu prüfen.
