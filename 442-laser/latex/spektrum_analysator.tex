\chapter{Spektrumanalysator}

Es soll nun die Modenstruktur der Laserkavität mittels eines Spektrumanalysator untersucht werden.
Bei dem Analysator handelt es sich um einen externen, konfokalen Resonator, dessen Länge mittels eines Piezo-Kristalls verstellt wird. \cite{Versuchsanleitung442}

Der Modenabstand von TEM\footnote{Tranversalen elektrischen Moden} bei einem solchen Resonator ist gegeben als: \cite{Versuchsanleitung442}

\begin{align}
	\Delta \nu_{TEM} = \frac{c}{4 l} \label{eq:mode_spacing_ext}
\end{align}

Bei der Laserkavität hingegen wird davon ausgegangen, dass in erster Linie Longitudinalmoden angeregt werden mit einem Modenabstand von:

\begin{align}
	\Delta \nu = \frac{c}{2L}
\end{align}

Nach der Justage des Analysators erfolgt die Messung indem ein Dreiecksignal an das Piezoelement angelegt wird, dessen Spannung so lange variiert wird, bis auf dem Oszilliskop eine periodische Struktur zu erkennen ist. Hierbei ist relevant, dass die Modenstruktur sowohl des Analysators als auch des Lasers überlagert beobachtet wird.

Eine kurze Überschlagsrechnung zeigt auf, dass der Modenabstand des Lasers ca. eine Größenordnung kleiner als bei dem Resonator zu erwarten ist. Somit kann davon ausgegangen werden, dass die näher aneinanderliegenden, feineren Maxima Lasermoden entsprechen und die größeren, weiter auseinanderliegenden Maxima entsprechen Analysatormoden sind.

Die Länge der Laserkavität ließ sich gut Messen, mit $L = \SI{51.3(5)}{\centi\meter}$. Die Länge des Analysators hingegen ist schwer akkurat zu messen, da die genaue Position der Spiegel in den jeweiligen Bauteilen nicht eindeutig ist und die Bauteile durch ihre Größe und Form ein genaues Messen erschweren. Es ergibt sich eine Länge von $l = \SI{20(15)}{\milli\meter}$
