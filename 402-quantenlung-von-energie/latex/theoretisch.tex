\chapter{Einleitung}

Die Wechselwirkung von elektromagnetischer Strahlung (Licht) und Materie findet grundsätzlich in quantisierten Einheiten und nicht kontinuierlich statt.
Dieses Phänomen wurde Anfang des 20. Jahrhunderts von \uppercase{Einstein} und \uppercase{Planck} theoretisch untersucht und ist heute eine wichtige Grundlage der modernen Physik \cite[62-65]{HakenWolf2013}.
In dem durchgeführten Versuch wird dieses quantisierte Verhalten an den Beispielen des äußeren photoelektrischen Effekt und dem Emissionsspektrum von Wasserstoffatomen untersucht.
Es werden der Wert des Planckschen Wirkungsquantums sowie die Wellenlängen der optisch sichtbaren Spektrallinien in der Balmer-Serie von Wasserstoff als Maß für die diskreten Energien des Atomanregungsstufen bestimmt.
Außerdem wird die Größe der Aufspaltung der Energieniveaus zwischen Wasserstoff und Deuterium als Wasserstoffisotop anhand der relativen Winkelaufspaltung der Spektrallinien der beiden Isotope untersucht.

Im ersten Versuchsteil wurde mithilfe der Gegenfeldmethode und einer Photozelle die Größe des Proportionalitätsfaktors des Planckschen Wirkungsquantums $h$ in der Energie-Wellenlängen-Beziehung
für Lichtquanten (Photonen) bestimmt, für die mit $\lambda$ als Wellenlänge und $\nu$ als Frequenz, gilt:
\begin{align}
	\label{eq:energielambda}
	E = h\nu = h \frac{c}{\lambda}
\end{align}
Für die verwendete Platin-Rhodium (Pt-Rh)-Anode wurde zudem die Austrittsarbeit der Elektronen aus den beobachten Strömen der Photozelle berechnet.


Im zweiten Versuchsteil wurde das optisch beobachtbare Spektrum der Emissionslinien einer Wasserstoff-Deuterium-Dampflampe
mithilfe eines Reflexionsgitter wellenlängenabhängig in verschiedene Winkel aufgespalten und mit dem Auge mithilfe von Linsen sowie einer CCD-Kamera vermessen,
sodass die den Wellenlängen zugeordneten Energien bestimmt werden können.
Das verwendete Gitter wurde dafür zudem noch mithilfe des bekannten Spektrums einer Quecksilber (Hg)-Lampe ausgemessen.


\chapter{Bestimmung des Planckschen Wirkungsquantums}

\section{Theoretische Überlegungen}

Wird monochromatisches Licht mit Frequenz $\nu$ auf eine Metallplatte gestrahlt, wechselwirken die Photonen mit den (Leitungs-)Elektronen des Metalls in dem sie ihre Energie $E=h\nu$ komplett an ein Elektron übertragen.
Übersteigt diese Energie die Austrittsarbeit, die für einen Austritt aus dem Material aufgebracht werden muss, können die Elektronen aus dem Material austreten
und der Elektronenstrom kann als Photostrom $I$ gemessen werden.
Ist die Energie der Photonen kleiner als die Austrittsarbeit kann kein Photostrom gemessen werden, unabhängig von der eingestrahlten Lichtintensität \cite[63]{HakenWolf2013}.
Dies ist der sogenannte äußere photoelektrischer Effekt (Photoeffekt).
Im Folgenden wird der Photoeffekt in einer Photozelle aus elektrisch verbundener Kathode,
die mit Licht bestrahlt wird mit Austrittsarbeit $W_K$, und Anode mit Austrittsarbeit $W_A$, an der der Photostrom gemessen wird, untersucht.
Es wird zudem eine Gegenspannung angelegt, die der Bewegung der Elektronen entgegenwirkt.
Bei der Gegenspannung $U_0$ (Grenzspannung), bei der gerade kein Photostrom mehr gemessen wird, d.h. keine Elektronen mehr die Anode erreichen, ergibt sich die Energiebilanz der Photoelektronen wie folgt:
Wenn Kathode und Anode verbunden werden, ergibt sich aufgrund der nun gleich großen Fermienergien $E_{\text{FK}}=E{_\text{FA}}$  der Leitungselektronen der beiden Materialien eine Kontaktspannung, welche ein Photoelektron überwinden muss:
\begin{align}
	\label{eq:kontaktspannung}
	e U_{\text{KA}} = \Delta W_{\text{Austritt}} = W_A - W_K
\end{align}
Bei angelegter Grenzspannung $U_0$ muss auch diese vom Elektron noch überwunden werden.
Die Bilanz ist für den Fall der angelegten Grenzspannung und unterschiedlich großen Austrittsarbeiten von Kathode und Anode aus verschiedenen Materialien,
wie sie im Experiment genutzt wurden, in Abbildung \ref{fig:energiebilanz_gegenspannung} graphisch dargestellt.
Zusammen ergibt sich für den Fall der Grenzspannung:
\begin{align}
	\label{eq:energiebilanz}
	 & E = h\nu = W_K + eU_{\text{KA}} + eU_0 \qquad | \quad eU_{\text{KA}} = W_A - W_K \nonumber \\
	 & \ergo h\nu = W_A + eU_0
\end{align}

\jafps{energiebilanz_gegenspannung}{Energiediagramm der Photozelle bei angelegter Gegenspannung $U_0$ und Kontaktspannung $eU_{\text{KA}}$. Grafik entnommen aus \cite[11]{Versuchsanleitung}.}{0.5}

Es kann also nur die Austrittsarbeit der Anode aus dem funktionellen Zusammenhang bestimmt werden, nicht die Austrittsarbeit der Kathode, deren Term sich in der Energiebilanz kürzt.

\subsection{Experimenteller Aufbau \& Durchführung}
Für die Messung des Photostroms wurde nach Justageanleitung in \cite[3-5]{Versuchsanleitung} auf einer optischen Bank als Lichtquelle eine Quecksilberdampflampe (Hg-Lampe) sowie
eine Irisblende zur Variation der Lichtintensität aufgebaut sowie eine dahinter folgende Sammellinse mit Brennweite $f=\SI{100}{\milli\meter}$.
Im Abstand $f$ wurde die Photozelle platziert, die hier aus einer Kalium(K)-Kathode und Pt-Rh-Anode besteht
(letzteres als Ringanode, damit möglichst kein Quellenlicht auf die Anode gelangt und dort ebenfalls Elektronen auslösen könnte), sodass das Licht scharf als Lichtfleck mit Durchmesser $(5-10)~\unit{\milli\meter}$ auf die Kathode fiel.
Zudem wurde noch ein Filterrad mit einstellbarer Durchlasswellenlänge vor die Photozelle platziert. Als Durchlasswellenlängen des Filterrads waren wählbar $\lambda \in \{365, 405, 436, 546, 578\}~\unit{\nano\meter}$. Um Streulicht zu vermeiden, wurde noch ein Rohr vor die quellenseitige Öffnung der Photozelle und mit kleinem Abstand zum Filterrad aufgesetzt.


%das hier kann wahrscheinlich noch bisschen schöner formuliert werden
Es wird der Anodenanschluss mit dem Netzgerät, welches Spannung der Stärke $(0-12)~\unit{\V}$ liefert, verbunden, sowie der Kathodenanschluss mit einem Messverstärker, welcher anschließend mit dem anderen Anschluss des Netzgeräts verbunden wurde.
Der Messverstärker wurde im Experiment auf die Stufe $\SI{e-10}{\A}\rightarrow \SI{1}{\V}$ gestellt.
Außerdem wurden zwei Digital-Multimeter zur Messung des Gegenstroms $U_G$ und Photostroms $I$, letzteren mittels der gemessenen Spannung
nach dem Messverstärker $U_I$, wie auf der Aufbauskizze in Abbildung \ref{fig:aufbau_photozelle} dargestellt, angeschlossen.

\jafps{aufbau_photozelle}{Aufbau zur Messung des Photostroms mit (von links nach rechts) Lichtquelle, Irisblende, Linse, Filterrad, Photozelle und Messaufbau. Grafik entnommen aus \cite[3]{Versuchsanleitung}}{0.6}

Zuletzt wurde noch eine Spannungsteilerschaltung zwischen die $\SI{12}{\V}$-Spannungsquelle und die Anode eingesetzt. %assuming die schaltskizze ist unnötig
Dafür wurden die beiden Widerstände $\SI{100}{\ohm}$ und $\SI{330}{\ohm}$ genutzt, wobei über den kleineren Widerstand die Spannung für die nachfolgende Schaltung abgegriffen wurde. Durch die niedrigere maximale Gegenspannung konnte der gesamte Bereich, in dem ein Photostrom floss, genauer vermessen werden.
Es ergab sich so für den Aufbau eine theoretische maximale Gegenspannung von
\(
U_{\text{max,th}} = \SI{12}{\V}\cdot \frac{\SI{100}{\ohm}}{\SI{430}{\ohm}} \approx\SI{2.80}{\V}.
\)
Es wurde mit dem Digital-Multimeter eine damit übereinstimmende tatsächlich verfügbare Maximalspannung von $U_{\text{max}}=\SI{-2.8(1)}{V}$ gemessen. Das Vorzeichen ergibt sich hier lediglich aus der Polung der Schaltung.
Bei dieser höchsten Spannung konnte selbst bei dem energiereichsten Licht, also bei der kürzesten Durchlasswellenlänge $\lambda = \SI{365}{\nano\meter}$, kein Photostrom mehr beobachtet werden, sondern nur noch ein sehr geringer Anodenstrom $I_0$ von Elektronen,
die sich aus der Anode lösen und sich zur Kathode bewegen, also einen positiven Strom darstellen.

Für jede der Durchlasswellenlängen wurde die folgende Messung jeweils zweimal durchgeführt:
Eine Wellenlänge des Filterrads wurde eingestellt, die Gegenspannung auf den maximalen Wert $U_{\text{max}}$ gestellt und die Größe des Anodenstroms $I_0$, bzw. die wirkende Dunkelspannung $U_{\text{I,0}}$ bestimmt.
Anschließend wurde die Kennlinie der Photozelle vollständig vermessen, indem die Gegenspannung über den gesamten Bereich $U_G=(0-(2,8\pm 0,1)\unit{\V}$ variiert und die Spannung $U_I$ gemessen wurde.
Durch die doppelte Messung jeder Kennlinie sollten Einflüsse von möglichen Intensitätsschwankungen der Hg-Lampe minimiert werden, welche den maximalen Photostrom bei kleiner Gegenspannung
durch eine Variation der vorhandenen Lichtquanten zeitlich beeinflusst hätten.
Für die Wellenlänge $\lambda = \SI{365}{\nano\meter}$ wurde anschließend die Blendenöffnung weiter geöffnet und so die Intensität des Lichts erhöht, sodass bei ausgestellter Gegenspannung ($U_G=\SI{0}{\V}$) eine um etwa Faktor $1,5$ größere Spannung an der Anode gemessen werden konnte.
Die Kennlinie der Photozelle wurde anschließend für diese Wellenlänge ein drittes Mal gemessen.


\section{Auswertung}

Aus den gemessenen Kennlinien werden nun Werte für das Planksche Wirkungsquantum $h$ und die Anoden-Austrittsarbeit $W_A$ bestimmt.

Aus der Messung des Photostroms $I$, korrigiert mit einem möglicherweise vorhandenen Anodenstrom $I_0$, der von einem minimalen Wert $U_0$ invers quadratisch mit dem Verlauf der angelegten Gegenspannung $U_G$ bis zu einem Sättigungsstrom $I_S$ steigt,
kann durch Interpolation der gemessenen Werte als x-Achsenabschnitt einer Darstellung von $\sqrt{(I-I_0)}(U_G)$ die Grenzspannung für Licht einer festen Wellenlänge bestimmt werden.

In der Energiebilanz für Photoelektronen bei der Grenzspannung gilt \footnote{mit ${e=\SI{1.602177e-19}{\C}}$ als Elementarladung\cite{KonstantenNIST}}:
\begin{align}
	\label{eq:gerade_u_gegen_h}
	U_0 = \frac{h}{e}\nu - \frac{W_A}{e} = \frac{h}{e}\frac{c}{\lambda} - \frac{W_A}{e}
\end{align}
Aus der Auftragung der Grenzspannungen $U_0$, die aus den Messungen der einzelnen Wellenlängen bestimmt wurden, gegen die Wellenlängen $\lambda$ können mithilfe einer Geradenanpassung $U_0 = a \lambda + b$ dann
der Wert des Plankschen Wirkungsquantums $h=\frac{a}{e}$ sowie die Austrittsarbeit der verwendeten Pt-Rh-Anode $W_A=-b \cdot e$ berechnet werden.



\section{Diskussion}
