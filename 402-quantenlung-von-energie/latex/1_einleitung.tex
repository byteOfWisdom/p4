\chapter{Einleitung}

Die Wechselwirkung von elektromagnetischer Strahlung (Licht) und Materie findet grundsätzlich in quantisierten Einheiten 
und nicht kontinuierlich statt.
Dieses Phänomen wurde Anfang des 20. Jahrhunderts von \uppercase{Einstein} und \uppercase{Planck} theoretisch untersucht und ist heute eine wichtige Grundlage der modernen Physik \cite[62-65]{HakenWolf2013}.
In dem durchgeführten Versuch wird dieses quantisierte Verhalten an den Beispielen des äußeren photoelektrischen Effekt und dem Emissionsspektrum von Wasserstoffatomen untersucht.
Es werden der Wert des Planckschen Wirkungsquantums sowie die Wellenlängen der optisch sichtbaren Spektrallinien in der Balmer-Serie von Wasserstoff als Maß für die diskreten Energien des Atomanregungsstufen bestimmt.
Außerdem wird die Größe der Aufspaltung der Energieniveaus zwischen Wasserstoff und Deuterium als Wasserstoffisotop anhand der relativen Winkelaufspaltung der Spektrallinien der beiden Isotope untersucht.

Im ersten Versuchsteil wurde mithilfe der Gegenfeldmethode und einer Photozelle die Größe des Proportionalitätsfaktors des Planckschen Wirkungsquantums $h$ in der Energie-Wellenlängen-Beziehung für Lichtquanten (Photonen) bestimmt, für die mit $\lambda$ als Wellenlänge und $\nu$ als Frequenz, gilt:
\begin{align}
	\label{eq:energielambda}
	E = h\nu = h \frac{c}{\lambda}
\end{align}
Für die verwendete Platin-Rhodium (Pt-Rh)-Anode wurde zudem die Austrittsarbeit der Elektronen aus den beobachten Strömen der Photozelle berechnet.


Im zweiten Versuchsteil wurde das optisch beobachtbare Spektrum der Emissionslinien einer Wasserstoff-Deuterium-Dampflampe mithilfe eines Reflexionsgitter wellenlängenabhängig in verschiedene Winkel aufgespalten und mit dem Auge mithilfe von Linsen sowie einer CCD-Kamera vermessen, sodass die den Wellenlängen zugeordneten Energien bestimmt werden können. Das verwendete Gitter wurde dafür zudem mithilfe des bekannten Spektrums einer Quecksilber (Hg)-Lampe ausgemessen.

%maybe bisschen mehr detail schon hier zu spektrallinien <-> energiezustände
